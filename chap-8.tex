% !Tex program = pdflatex
% 第 8 章: 特征值和特征向量 Chap 8: Eigenvalues and Eigenvectors
\ifx\allfiles\undefined
\documentclass{note}
\setcounter{chapter}{+7}
\begin{document}
\fi
\chapter{特征值和特征向量}
在上一章中, 我们介绍了多项式对应的伴阵, 那么, 如何由伴阵恢复多项式呢?

假设多项式 $p(x)=x^2+ax+b$, 则其伴阵为 $C[p(x)]=\begin{pmatrix}
    0&-b\\
    1&-a
\end{pmatrix}$,\\
$\det(xI-C[p(x)])=\begin{vmatrix}
    x&b\\
    -1&x+a
\end{vmatrix}=x^2+ax+b$ 即恢复多项式.\\
同理, 对循环不变子空间 $\langle\langle v_{ij}\rangle\rangle$, 设其极小多项式为 $p_i^{e_{ij}}(x)$, 伴阵为 $C[p_i^{e_{ij}}(x)]$, 有 $\det(xI-C[p_i^{e_{ij}}])=p_i^{e_{ij}}(x)$,\\
由定理 \ref{thm-7.10}, 线性算子 $\tau$ 的表示即为这些伴阵的直和, 故有
\begin{df}[特征多项式]
    $\det(xI-[\tau]_{\mathcal{B}})=\prod_{ij}p_i^{e_{ij}}(x)\equiv C_{\tau}(x)$, 称为 $\tau$ 的\textbf{特征多项式}.
\end{df}

\begin{thm}[(课本第 3 版 引理 7.17, 定理 7.18)]
    特征多项式在相似操作下不变, 或线性算子的特征多项式唯一, 与线性算子的表示无关.
\end{thm}
\begin{pf}
    设线性算子 $\tau$ 在定序基 $\mathcal{B}$ 和 $\mathcal{B}'$ 下的表示分别为 $[\tau]_{\mathcal{B}}$ 和 $[\tau]_{\mathcal{B}'}$,\\
    则 $C_{[\tau]_{\mathcal{B}'}}(x)=\det(xI-[\tau]_{\mathcal{B}'})=\det(xI-M_{\mathcal{BB}'}[\tau]_{\mathcal{B}}M_{\mathcal{BB}'}^{-1})=\det(M_{\mathcal{BB}'}(xI-[\tau]_{\mathcal{B}})M_{\mathcal{BB}'}^{-1})=\det(M_{\mathcal{BB}'}^{-1}M_{\mathcal{BB}'}(xI-[\tau]_{\mathcal{B}}))=\det(xI-[\tau]_{\mathcal{B}})=C_{[\tau]_{\mathcal{B}}}(x)$.
\end{pf}

$\because C_{\tau}(x)=\prod_{ij}p_i^{e_{ij}}=\prod_ip_i^{\sum_{j=1}^{t_i}e_{ij}}$, $m_{\tau}(x)=up_1^{e_1}\cdots p_m^{e_m}$, 且 $e_i=e_{i1}\geq\cdots\geq e_{it_i}$ (不可约因子相同, 但幂次不等), $\therefore\deg C_{\tau}(x)=\sum_{ij}\deg p_i^{e_{ij}}(x)=\sum_{ij}d_{ij}=n\geq\deg m_{\tau}(x)\Longrightarrow m_{\tau}(x)\mid C_{\tau}(x)$.


在 $F[x]$ 中, 任一一次多项式 $x-r$ 都是不可约的.\\
在 $\mathbb{R}[x]$ 中, 有实根 $\Longleftrightarrow$ 可约.\\
$\because$ 实系数多项式的复根的共轭亦为该多项式的根, $\therefore$ 实系数多项式的复根总是成对出现\footnote{\begin{pf}
    $0\neq f(x)\in\mathbb{R}[x]$, 若 $z\in\mathbb{C}$, s.t. $f(z)=0$, 即 $z^n+a_{n-1}z^{n-1}+\cdots+a_1z+a_0=0$\\
    $\Longrightarrow\overline{f(z)}=\bar{z}^n+a_{n-1}\bar{z}^{n-1}+\cdots+a_1\bar{z}+a_0=f(\bar{z})=0$, 故得证.
\end{pf}}, 故 $\mathbb{R}[x]$ 中的不可约多项式阶数 $\leq 2$.
\begin{pf}
    当多项式阶数 $>2$, 若阶数为奇数, 则多项式的根两两配对后必留下一个根, 该根必不为复数而为实数 $\Longrightarrow$ 可约;\\
    若阶数为偶数, 以 $4$ 阶为例, 假设多项式不可约, 设复根分别为 $z_1,\bar{z_1},z_2,\bar{z_2}$, 则多项式可写为 $(x-z_1)(x-\bar{z}_1)(x-z_2)(x-\bar{z}_2)=(x-(z_1+\bar{z}_1)x+z_1\bar{z}_1)(x-(z_2+\bar{z}_2)x+z_2\bar{z}_2)$.\\
    $(z_1+\bar{z}_1),z_1\bar{z_1},(z_2+\bar{z}_2),z_2\bar{z_2}\in\mathbb{R}$, $\therefore (x-(z_1+\bar{z}_1)x+z_1\bar{z}_1),(x-(z_2+\bar{z}_2)x+z_2\bar{z}_2)\in\mathbb{R}[x]\Longrightarrow$ 多项式可约, 故假设错误, 阶数 $>2$ 的偶数阶实系数多项式必可约.
\end{pf}
\begin{eg}
    $x^2+1=(x+i)(x-i)$ 无实根不可约.
\end{eg}
$\mathbb{Q}[x]$ 中的不可约多项式可以是任意次.\\
可用 \textbf{Eisenstein 方法} 判别 $\mathbb{Q}[x]$ 中的多项式的可约性: $\forall f(x)=a_0+a_1x+\cdots+a_nx^n$, 其中 $a_i\in\mathbb{Z}$, 若 $\exists$ 素数 $p$, s.t. $p\nmid a_n$, $p\mid a_i\vert_{i=0,\cdots,i=n-1}$, $p^2\nmid a_0$, 则 $f(x)$ 在 $\mathbb{Q}[x]$ 中不可约.

\begin{df}[特征值和特征向量]
    $\tau\in\mathcal{L}(V)$, $\lambda\in F$, 若 $\exists 0\neq v\in V$, s.t. $\tau(v)=\lambda v$, 则称 $\lambda$ 为 $\tau$ 的\textbf{特征值}, $v$ 为 $\tau$ 的\textbf{特征向量}.
\end{df}

\begin{df}
    $\mathcal{E}_{\lambda}\equiv\{\text{特征值 $\lambda$ 对应的特征向量}\}=\{v\in V\mid\tau(v)=\lambda v\}$.
\end{df}

$\mathcal{E}_{\lambda}$ 是 $V$ 的子空间.
\begin{pf}
    $\forall u,v\in\mathcal{E}_{\lambda}$, $\forall r,t\in F$, $\tau(ru+tv)=r\tau(u)+t\tau(v)=r(\lambda v)+t(\lambda v)=\lambda(ru)+\lambda(tv)=\lambda(ru+tv)\Longrightarrow ra+tv\in\mathcal{E}_{\lambda}$, 故得证.
\end{pf}

\begin{df}[特征谱]
    $\spec(\tau)\equiv\{\tau\text{ 的特征值}\}$.
\end{df}

\begin{itemize}
    \item[(1)] $\tau(v)=\lambda v\Longleftrightarrow\tau(v)-\lambda v=0$\\
    $\Longleftrightarrow(\tau-\lambda)(v)=0$\\
    $\Longleftrightarrow v\in\ker(\tau-\lambda)$\\
    ($\because v\neq 0$)\\
    $\Longleftrightarrow\ker(\tau-\lambda)\neq\emptyset$\\
    $\Longleftrightarrow(\tau-\lambda)$ 非单\\
    $\Longleftrightarrow(\tau-\lambda)$ 是退化的.\\
    (也有教材将 $\tau-\lambda$ 退化作为特征值的定义.)
    \item[(2)] $v\neq 0$ 是 $\lambda$ 对应的特征向量, $\tau(v)=\lambda v\Longleftrightarrow(\tau-\lambda)(v)=0$.\\
    若 $f(x)=x-\lambda$, 则 $f(\tau)(v)=0$, 即 $f(x)$ 零化 $v\Longrightarrow f(x)\in\ann(v)$.\\
    又 $\because x-\lambda$ 不可约, $\therefore\ann(v)=\langle x-\lambda\rangle$.\\
    $\because\ann(V)=\langle m_{\tau}(x)\rangle$, $\therefore\forall u\in V$, $m_{\tau}(x)u=0\Longrightarrow m_{\tau}(x)v=0\Longrightarrow m_{\tau}(x)\in\ann(v)=\langle x-\lambda\rangle\Longrightarrow(x-\lambda)\mid m_{\tau}(x)$.\\
    设 $m_{\tau}(x)=(x-\lambda)q(x)$, 其中 $q(x)\in F[x]$, 故特征值为极小多项式的根.
\end{itemize}

\begin{thm}[(课本定理 8.3)]
    \begin{itemize}
        \item[(1)] $\spec(\tau)$ 是 $m_{\tau}(x)$ 或 $C_{\tau}(x)$ 的根的集合.
        \item[(2)] 矩阵的特征值在相似操作下不变, 即相似矩阵具有相同的特征值.
        \item[(3)] $\mathcal{E}_{\lambda}$ 是 $(\lambda I-A)x=0$ 的解空间.
    \end{itemize}
\end{thm}

\begin{thm}[(课本定理 8.4)]
    $\tau\in\mathcal{L}(V)$, $\lambda_1,\cdots,\lambda_k$ 是 $\tau$ 的特征值且互不相等, 则
    \begin{itemize}
        \item[(1)] 取 $0\neq v_i\in\mathcal{E}_{\lambda_i}$, 则 $\{v_1,\cdots,v_k\}$ 线性无关;
        \item[(2)] 若 $i\neq j$, $\mathcal{E}_{\lambda_i}\cap\mathcal{E}_{\lambda_j}=\{0\}$.
    \end{itemize}
\end{thm}
\begin{pf}
    \begin{itemize}
        \item[(1)] 若 $\sum_{i=1}^kr_iv_i=0$, 则 $0=\tau(0)=\tau\left(\sum_{i=1}^kr_iv_i\right)=\sum_{i=1}^kr_i\tau(v_i)=\sum_{i=1}^kr_i\lambda_iv_i$, 即 $r_1\lambda_1v_1+\cdots+r_k\lambda_kv_k=0$,\\
        又有 $\lambda_k\left(\sum_{i=1}^kr_iv_i\right)=0$, 即 $r_1\lambda_kv_1+r_2\lambda_kv_2+\cdots+r_k\lambda_kv_k=0$.\\
        以上两式相减消去 $v_k$ 项得, $r(\lambda_1-\lambda_k)v_1+r_2(\lambda_2-\lambda_k)v_2+\cdots+r_{k-1}(\lambda_{k-1}-\lambda_k)v_{k-1}=0$.\\
        将 $\tau$ 作用于上式得, $\tau\left[r(\lambda_1-\lambda_k)v_1+r_2(\lambda_2-\lambda_k)v_2+\cdots+r_{k-1}(\lambda_{k-1}-\lambda_k)v_{k-1}\right]=r_1(\lambda_1-\lambda_k)\lambda_1v_1+r_2(\lambda_2-\lambda_k)\lambda_2v_2+\cdots+r_{k-1}(\lambda_{k-1}-\lambda_k)\lambda_{k-1}v_{k-1}=\tau(0)=0$.\\
        消去 $v_{k-1}$ 项得, $\lambda_{k-1}\left[r(\lambda_1-\lambda_k)v_1+r_2(\lambda_2-\lambda_k)v_2+\cdots+r_{k-1}(\lambda_{k-1}-\lambda_k)v_{k-1}\right]-\tau\left[r(\lambda_1-\lambda_k)v_1+r_2(\lambda_2-\lambda_k)v_2+\cdots+r_{k-1}(\lambda_{k-1}-\lambda_k)v_{k-1}\right]=r_1(\lambda_k-\lambda_1)(\lambda_{k-1}-\lambda_1)v_1+r_2(\lambda_k-\lambda_2)(\lambda_{k-1}-\lambda_2)v_2+\cdots+r_{k-1}(\lambda_k-\lambda_{k-2})(\lambda_{k-1}-\lambda_{k-2})v_{k-2}=0$.\\
        重复以上操作 $k-1$ 次得, $r_1(\lambda_2-\lambda_1)\cdots(\lambda_k-\lambda_1)v_1=0$.\\
        $\because\lambda_1,\cdots,\lambda_k$ 互不相等且 $v_1\neq 0$, $\therefore r_1=0$.\\
        同理可得 $r_1=\cdots=r_k=0$, 故 $\{v_1,\cdots,v_k\}$ 线性无关.
        \item[(2)] 取 $u\in\mathcal{E}_{\lambda_i}\cap\mathcal{E}_{\lambda_j}\Longrightarrow u\in\mathcal{E}_{\lambda_i}\Longrightarrow\tau(u)=\lambda_iu$,\\
        且 $u\in\mathcal{E}_{\lambda_j}\Longrightarrow\tau(u)=\lambda_ju$.\\
        以上两式相减得, $0=\tau(u)-\tau(u)=\lambda_iu-\lambda_ju=(\lambda_i-\lambda_j)u$.\\
        又 $\because\lambda_i\neq\lambda_j$, $\therefore u=0\Longrightarrow\mathcal{E}_{\lambda_i}\cap\mathcal{E}_{\lambda_j}=\{0\}$.
    \end{itemize}
\end{pf}

若 $\tau\in\mathcal{L}(V)$, $p(x)\in F[x]$, 则 $p(\tau)\in\mathcal{L}(V)$, 那么 $p(\tau)$ 与 $\tau$ 的特征值有何关系?

\begin{thm}[特征谱映射定理 (课本第 3 版定理 8.3)]
    $F$ 为代数闭域\footnote{即 $F[x]$ 中不可约多项式阶数 $=1$.}, $p(x)\in F[x]$, 则 $\spec(p(\tau))=p(\spec(\tau))\equiv\{p(\lambda)\mid\lambda\in\spec(\tau)\}$.
\end{thm}
\begin{pf}
    先证 $p(\spec(\tau))\subseteq\spec(p(\tau))$: $\forall v\in\spec(\tau)$, $\tau(v)=\lambda v\Longrightarrow\tau^k(v)=\lambda^kv$.\\
    设 $p(x)=x^d+a_{d-1}x^{d-1}+\cdots+a_1x+a_0$, 则 $p(\tau)=\tau^d+a_{d-1}\tau^{d-1}+\cdots+a_1\tau+a_0$.\\
    $p(\tau)(v)=\tau^d(v)+a_{d-1}\tau^{d-1}(v)+\cdots+a_1\tau(v)+a_0=\lambda^dv+a_{d-1}\lambda^{d-1}v+\cdots+a_1\lambda v+a_0=(\lambda^d+a_{d-1}\lambda^{d-1}+\cdots+a_1\lambda+a_0)v=p(\lambda)v$\\
    $\Longrightarrow p(\lambda)$ 是 $p(\tau)$ 的特征值, 即 $p(\lambda)\in\spec(p(\tau))$, 故 $p(\spec(\tau))\subseteq\spec(p(\tau))$.

    再证 $\spec(p(\tau))\subseteq p(\spec(\tau))$: $\forall\lambda\in\spec(p(t))$, $0\neq v\in\mathcal{E}_{\lambda}$, 即 $p(\tau)(v)=\lambda v\Longrightarrow(p(\tau)-\lambda)(v)=0$.\\
    令 $f(x)=p(x)-\lambda$, 则 $f(\tau)(v)=0$.\\
    $\because\lambda\in F$, $\therefore f(x)\in F[x]$.\\
    $\because F[x]$ 是代数封闭的, $\therefore f(x)$ 可写为 $f(x)=(x-r_1)\cdots(x-r_m)$, 其中 $r_i\in F$, 或有重复\\
    $\Longrightarrow(\tau-r_1)\cdots(\tau-r_m)(v)=0$, 其中 $r_i\in\spec(\tau)$,\\
    其中必 $\exists r_i$, s.t. $(\tau-r_i)\cdots(\tau-r_m)(v)=0\Longrightarrow p(r_i)-\lambda=0\Longrightarrow p(r_i)=\lambda$\\
    $\Longrightarrow\lambda\in p(\spec(\tau))$, 故 $\spec(p(\tau))\subseteq p(\spec(\tau))$.

    综上, 得证.
\end{pf}

\begin{df}[约当标准型]
    $F$ 为代数闭域, $V$ 为 $F$ 上的向量空间, $\dim V<\infty$, $\tau\in\mathcal{L}(V)$, 则 $V$ 的最小多项式 $m_{\tau}(x)=(x-\lambda_1)^{e_1}\cdots(x-\lambda_m)^{e_m}$,\\
    此时 $V=\bigoplus_{i=1}^mV_i$, 其中 $\ann(V_i)=\langle(x-\lambda_i)^{e_i}\rangle$,\\
    $V_i=\bigoplus_{j=1}^{t_i}\langle\langle v_{ij}\rangle\rangle$, 其中 $\ann(v_{ij})=\langle(x-\lambda_i)^{e_{ij}}\rangle$, $e_i=e_{i1}\geq\cdots\geq e_{it_i}$, $\dim\langle\langle v_{ij}\rangle\rangle=\deg(x-\lambda_i)^{e_{ij}}=e_{ij}$,\\
    则定序基 $\mathcal{B}_{ij}'=\{v_{ij},(\tau-\lambda_i)(v_{ij}),\cdots,(\tau-\lambda_i)^{e_{ij}-1}(v_{ij})\}$ 为 $\langle\langle v_{ij}\rangle\rangle$ 下, 线性算子 $\tau$ 在 $\langle\langle v_{ij}\rangle\rangle$ 中可表为 $[\tau]_{\mathcal{B}_{ij}'}=\begin{pmatrix}
        [\tau(b_1)]_{\mathcal{B}_{ij}'}&\cdots&[\tau(b_m)]_{\mathcal{B}_{ij}'}
    \end{pmatrix}$,\\
    其中 $[\tau(b_1)]_{\mathcal{B}_{ij}'}=[\tau(v_{ij})]_{\mathcal{B}_{ij}'}=[(\tau-\lambda_i+\lambda_i)(v_{ij})]_{\mathcal{B}_{ij}'}=[(\tau-\lambda_i)(v_{ij})+\lambda_i(v_{ij})]_{\mathcal{B}_{ij}'}=\begin{pmatrix}
        \lambda_i\\
        1\\
        0\\
        \vdots\\
        0
    \end{pmatrix}$, $[\tau(b_2)]_{\mathcal{B}_{ij}'}=[\tau(\tau-\lambda_i(v_{ij}))]_{\mathcal{B}_{ij}'}=[(\tau-\lambda_i+\lambda_i)((\tau-\lambda_i)(v_{ij}))]_{\mathcal{B}_{ij}'}=[(\tau-\lambda_i)^2(v_{ij})+\lambda_i(\tau-\lambda_i)(v_{ij})]_{\mathcal{B}_{ij}'}=\begin{pmatrix}
        0\\
        \lambda_i\\
        1\\
        0\\
        \vdots\\
        0
    \end{pmatrix}$, $\cdots$, $[\tau((\tau-\lambda_i)^{e_{ij}-1}(v_{ij}))]=[(\tau-\lambda_i+\lambda_i)((\tau-\lambda_{ij})^{e_{ij}-1}(v_{ij}))]_{\mathcal{B}_{ij}'}=[(\tau-\lambda_i)^{e_{ij}}(v_{ij})+\lambda_i(\tau-\lambda_i)^{e_{ij}-1}(v_{ij})]_{\mathcal{B}_{ij}'}=[\lambda_i(\tau-\lambda_i)^{e_{ij}-1}(v_{ij})]_{\mathcal{B}_{ij}'}=\begin{pmatrix}
        0\\
        \vdots\\
        0\\
        \lambda_i
    \end{pmatrix}$,\\
    从而 $[\tau]_{\mathcal{B}_{ij}'}=\begin{pmatrix}
        \lambda_i&0&\cdots&\cdots&0&0\\
        1&\lambda_i&\cdots&\cdots&0&0\\
        0&1&\cdots&\cdots&0&0\\
        \vdots&\vdots&\ddots&\ddots&\vdots&\vdots\\
        0&0&\cdots&\cdots&\lambda_i&0\\
        0&0&\cdots&\cdots&1&\lambda_i
    \end{pmatrix}\equiv g(\lambda_i,e_{ij})$, 称为\textbf{约当块},\\
    定序基 $\mathcal{B}=\cup_{ij}\mathcal{B}_{ij}'$ 下, 线性算子 $\tau$ 可表为
    $$
    [\tau]_{\mathcal{B}}=\begin{pmatrix}
        g(\lambda_1,e_{11})\\
        &\ddots\\
        &&g(\lambda_1,e_{1t_1})\\
        &&&\ddots\\
        &&&&g(\lambda_m,e_{m1})\\
        &&&&&\ddots\\
        &&&&&&g(\lambda_m,e_{mt_m})
    \end{pmatrix},
    $$
    称为\textbf{约当标准型}.\\
    \footnote{$\mathcal{B}_{ij}=\{v_{ij},\tau(v_{ij}),\cdots,\tau^{e_{ij}-1}(v_{ij})\}$ 亦为 $\langle\langle v_{ij}\rangle\rangle$ 的一组基, $[\tau]_{\mathcal{B}=\cup_{ij}\mathcal{B}_{ij}}$ 为有理标准型.}
\end{df}

$\mathcal{B}_{ij}'=\{v_{ij},(\tau-\lambda_i)(v_{ij}),\cdots,(\tau-\lambda_i)^{e_{ij}-1}(v_{ij})\}$ 为 $\langle\langle v_{ij}\rangle\rangle$ 的一组基.
\begin{pf}
    先证 $\mathcal{B}_{ij}'$ 线性无关: 设 $l_0v+l_1(\tau-\lambda_i)(v)+\cdots+l_{m-1}(\tau-\lambda_i)^{m-1}(v)=0$.\\
    令 $h(x)=l_0+l_1x+\cdots+l_{m-1}x^{m-1}$, 则 $h(x)v=0\Longrightarrow h(x)\in\ann(v_{ij})=\langle(x-\lambda_i)^{e_{ij}}\rangle\Longrightarrow(x-\lambda_i)^{e_{ij}}\mid h(x)$.\\
    然而 $\because\deg(x-\lambda_i)^{e_{ij}}=e_{ij}\geq\deg h(x)=e_{ij}-1$, $\therefore l_0=l_1=\cdots=l_{e_{ij}-1}=0$, 故 $\mathcal{B}_{ij}'$ 线性无关.

    再证 $\mathcal{B}_{ij}'$ 生成 $\langle\langle v_{ij}\rangle\rangle$: $\langle\langle v_{ij}\rangle\rangle=\{h(x)v\mid h(x)\in F[x]\}=\{h(\tau)(v)\mid h(x)\in F[x]\}$.\\
    $\forall h(\tau)(v)\in\langle\langle v_{ij}\rangle\rangle$, 若 $\deg h(x)\leq e_{ij}-1$, 则 $h(\tau)v=(l_0+l_1\tau+\cdots+l_{e_{ij}-1}\tau^{e_{ij}-1})(v)=(l_0+l_1((\tau-\lambda_i)+\lambda_i)+\cdots+l_{e_{ij}-1}((\tau-\lambda_i)+\lambda_i)^{e_{ij}-1})(v)$, 可由 $\mathcal{B}_{ij}'$ 表示,\\
    若 $\deg h(x)>e_{ij}-1$, $h(x)=h((x-\lambda_i)+\lambda_i)=h'(x-\lambda_i)=q(x-\lambda_i)(x-\lambda_i)^{e_{ij}-1}+r(x-\lambda_i)$, 其中 $q(x-\lambda_i)$ 为商多项式, 余多项式 $r(x-\lambda_i)=0$ 或 $\deg r(x)<\deg(x-\lambda_{ij})^{e_{ij}-1}=e_{ij}-1$\\
    $\Longrightarrow h(\tau)(v_{ij})=h'(\tau-\lambda_i)(v_{ij})=(q(\tau-\lambda_i)(\tau-\lambda_i)^{e_{ij}-1}+r(\tau-\lambda_i))(v_{ij})=q(\tau-\lambda_i)(\tau-\lambda_i)^{e_{ij}-1}(v_{ij})+r(\tau-\lambda_i)(v_{ij})$, 其中 $\because(x-\lambda_i)^{e_{ij}-1}\in\ann(v_{ij})$, $\therefore(\tau-\lambda_i)^{e_{ij}-1}(v_{ij})=0\Longrightarrow h(\tau)v_{ij}=r(\tau)(v_{ij})$, 可由 $\mathcal{B}_{ij}'$ 表示.

    综上, 得证.
\end{pf}

有理标准型的存在无需附加条件, 而约当标准型的存在是需要附加条件的: 仅当极小多项式能分解成一次多项式的乘积, 即 $m_{\tau}(x)=(x-\lambda_1)^{e_1}\cdots(x-\lambda_m)^{e_m}$ 时, 约当标准型才存在.

上面我们看到, $\tau((\tau-\lambda_i)^{e_{ij}-1}(v_{ij}))=\lambda(\tau-\lambda_i)^{e_{ij}}(v_{ij})$, 即 $(\tau-\lambda_i)^{e_{ij}-1}(v_{ij})\in\mathcal{E}_{\lambda_i}$, 那么, 是否还有其他向量 $\in\mathcal{E}_{\lambda_i}$?

$\{(\tau-\lambda_i)^{e_{ij}-1}(v_{ij})\mid j=1,\cdots,t_i\}\subseteq\mathcal{E}_{\lambda_i}$, $\because V_{p_i}=\bigoplus_{j=1}^{t_i}\langle\langle v_{ij}\rangle\rangle$, $\therefore\{(\tau-\lambda_i)^{e_{ij}-1}(v_{ij})\mid j=1,\cdots,t_i\}$ 线性无关.

\begin{df}[代数重数]
    在 $\mathcal{C}_{\tau}(x)$ 中 $\lambda_i$ 作为根的重数, 即 $\dim V_{p_i}=\sum_{j=1}^{t_i}e_{ij}$.
\end{df}

\begin{df}[几何重数]
    $\dim\mathcal{E}_{\lambda_i}=t_i$.
\end{df}

\begin{thm}[(课本定理 8.5)]
    几何重数 $\dim V_{p_i}=\sum_{j=1}^{t_i}e_{ij}\geq$ 代数重数 $\dim\mathcal{E}_{\lambda_i}=t_i$
\end{thm}
\begin{pf}
    $e_i\geq e_{i1}\geq\cdots\geq e_{it_i}\geq 0$, 故得证.
\end{pf}

几何重数 $=$ 代数重数的特殊情况下, $[\tau]$ 的约当标准型何如?

若几何重数 $=$ 代数重数, 即 $\dim V_{p_i}=\dim\mathcal{E}_{\lambda_i}$, 则 $e_{ij}=1\forall j\Longrightarrow m_{\tau}(\lambda)=(x-\lambda_1)\cdots(x-\lambda_m)$.\\
此时 $[\tau]_{\mathcal{B}_{ij}'}=\begin{pmatrix}
    \lambda_i
\end{pmatrix}$, $[\tau]_{\mathcal{B}'}=\diag(\lambda_1,\cdots,\lambda_m)$.

\begin{thm}[(课本定理 8.10, 8.11, 8.18)]\label{thm-8.10-8.11-8.18}
    下列叙述等价:
    \begin{itemize}
        \item[(1)] $\tau$ 可对角化, 即 $\exists$ 一组基 $\mathcal{B}$, s.t. $[\tau]_{\mathcal{B}}$ 为对角阵.
        \item[(2)] 几何重数 $=$ 代数重数.
        \item[(3)] $m_{\tau}(x)=(x-\lambda_1)\cdots(x-\lambda_m)$, $\lambda_i$ 互不相等.
        \item[(4)] $V_{p_i}=\mathcal{E}_{\lambda_i}$, $V=\mathcal{E}_{\lambda_1}\oplus\cdots\oplus\mathcal{E}_{\lambda_m}$.
        \item[(5)] 特征向量构成 $V$ 的基.
        \item[(6)] $\tau=\lambda_1\rho_1+\cdots+\lambda_k\rho_k$, 其中 $\rho_1+\cdots+\rho_k=1$ 为单位分解, $\spec(\tau)=\{\lambda_1,\cdots,\lambda_k\}$, $\im\rho_i=\mathcal{E}_{\lambda_i}$, $\ker\rho_i=\bigoplus_{j\neq i}\mathcal{E}_{\lambda_j}$.
    \end{itemize}
\end{thm}

\section{投影算子}
\begin{df}[投影(算子)]
    向量空间 $V=S\oplus T$, 映射 $\rho_{ST}:V\rightarrow V$, $u_S+u_T\mapsto u_S$, 其中 $u_S\in S$, $u_T\in T$, 则 $\rho_{ST}$ 称为在 $S$ 上沿 $T$ 的\textbf{投影(算子)}.
\end{df}

$\ker\rho_{ST}=T$, $\im\rho_{ST}=S$, $V=\ker\rho_{ST}\oplus\im\rho_{ST}$.

\begin{thm}[(课本第 3 版定理 2.21)]\label{thm-2.21}
    \begin{itemize}
        \item[(1)] $V=S\oplus T$, 则 $\rho_{ST}+\rho_{TS}=1_V$ ($V$ 上的恒等变换) 且 $\rho_{ST}\circ\rho_{TS}=\rho_{TS}\circ\rho_{ST}=0_V$ ($V$ 上的零变换).
        \item[(2)] $\sigma\in\mathcal{L}(V)$, 若 $V=\ker\sigma\oplus\im\sigma$ 且 $\sigma\mid_{\im\sigma}=1|_{\im\sigma}$, 其中 $|_{\im\sigma}$ 代表算子定义域为 $\im\sigma$, 则 $\sigma$ 是在 $\im\sigma$ 上沿 $\ker\sigma$ 的投影.
    \end{itemize}
\end{thm}
\begin{pf}
    \begin{itemize}
        \item[(1)] $\forall v\in V$, $\because V=S\oplus T$, $\therefore v=v_S+v_T$, 其中 $v_S\in S$, $v_T\in T$\\
        $\Longrightarrow(\rho_{ST}+\rho_{TS})(v)=(\rho_{ST}+\rho_{TS})(v_S+v_T)=\rho_{ST}(v_S)+\rho_{ST}(v_T)+\rho_{TS}(v_S)+\rho_{TS}(v_T)=v_S+0+0+v_T=v\Longrightarrow\rho_{ST}+\rho_{TS}=1$;\\
        $\Longrightarrow\rho_{ST}\circ\rho_{TS}(v)=\rho_{ST}(\rho_{TS}(v_S+v_T))=\rho_{ST}(\rho_{TS}(v_S)+\rho_{TS}(v_T))=\rho_{ST}(0+v_T)=0\Longrightarrow\rho_{ST}\circ\rho_{TS}=0$.\\
        同理, $\rho_{TS}\circ\rho_{ST}=0$.
        \item[(2)] $\forall v\in V$, $\because V=\ker\sigma\oplus\im\sigma$, $\therefore v=v_{\ker}+v_{\im}$, 其中 $v_{\ker}\in\ker\sigma$, $v_{\im}\in\im\sigma$\\
        $\Longrightarrow\sigma(v)=\sigma(v_{\ker}+v_{\im})=\sigma(v_{\ker})+\sigma(v_{\im})=\sigma(v_{\ker})+1(v_{\im})=0+v_{\im}=v_{\im}$, 故 $\sigma$ 为 $\im\sigma$ 上沿 $\ker\sigma$ 的投影.
    \end{itemize}
\end{pf}

\begin{thm}[(课本定理第 3 版 2.22)]
    $\rho\in\mathcal{L}(V)$ 为投影 $\Longleftrightarrow\rho^2=\rho$.
\end{thm}
\begin{pf}
    ``$\Longrightarrow$'': $\forall v\in V$, $v=u_S+u_T$, 其中 $u_S\in S$, $u_T\in T$.\\
    $\rho(v)=u_S$, $\rho(\rho(v))=\rho(u_S)=u_S\Longrightarrow\rho^2=\rho$.

    ``$\Longleftarrow$'': 首先将 $V$ 分解成 $V=\ker\rho\oplus\im\rho$:\\
    一方面, $\forall v\in\ker\rho\cap\im\rho\Longrightarrow v\in\ker\rho\Longleftrightarrow\rho(v)=0$,\\
    且 $v\in\im\rho\Longleftrightarrow\exists u\in V$, s.t. $v=\rho(u)$\\
    $\Longleftrightarrow 0=\rho(v)=\rho(\rho(u))=\rho^2(u)=\rho(u)=v\Longrightarrow\ker\rho\cap\im\rho=\{0\}$.\\
    另一方面, $\forall v\in V$, $\rho(v)\in\im\rho$.\\
    $\because\rho(v-\rho(v))=\rho(v)-\rho(\rho(v))=\rho(v)-\rho^2(v)=0$, $\therefore v=(v-\rho(v))+\rho(v)$, 其中 $v-\rho(v)\in\ker\rho$, $\rho(v)\in\im\rho\Longrightarrow V=\ker\rho+\im\rho$.\\
    故 $V=\ker\rho\oplus\im\rho$.

    又 $\because\forall\rho(u)\in\im\rho$, $\rho(\rho(u))=\rho^2(u)=\rho(u)\Longrightarrow\rho|_{\im\rho}=1|_{\im\rho}$, $\therefore$ 由定理 \ref{thm-2.21}, $\rho$ 为在 $\im\rho$ 上沿 $\ker\rho$ 的投影.

    综上, 得证.
\end{pf}

\begin{df}[正交]
    $\rho,\sigma\in\mathcal{L}(V)$ 为投影, 若 $\rho\circ\sigma=\sigma\circ\rho=0$, 则称 $\rho$ 与 $\sigma$ \textbf{正交}, 记作 $\rho\perp\sigma$.
\end{df}

$\rho\perp\sigma\Longleftrightarrow\forall v\in V$, $\rho\circ\sigma(v)=\rho(\sigma(v))=0$ 且 $\sigma\circ\rho(v)=\sigma(\rho(v))=0\Longleftrightarrow\im\sigma\subseteq\ker\rho$ 且 $\im\rho\subseteq\ker\sigma$.

\begin{df}[单位分解]
    $\rho_1+\cdots+\rho_k=1$, 其中 $\rho_i$ 为投影且互相 $\perp$, 则称该式为\textbf{单位分解}.
\end{df}

\begin{thm}[(课本第 3 版定理 2.25)]
    \begin{itemize}
        \item[(1)] $\rho_1+\cdots+\rho_k=1$ 为单位分解, 则 $\im\rho_1\oplus\cdots\oplus\im\rho_k=V$.
        \item[(2)] 若 $V=S_1\oplus\cdots\oplus S_k$, 令 $\rho_i$ 是在 $S_i$ 上沿 $\sum_{j\neq i}S_j$ 的投影, 则 $\rho_1+\cdots+\rho_k=1$ 为单位分解.
    \end{itemize}
\end{thm}
\begin{pf}
    \begin{itemize}
        \item[(1)] 先证 $V$ 由 $\{\im\rho_1,\cdots,\im\rho_k\}$ 生成: $\forall v\in V$, $v=1(v)=(\rho_1+\cdots+\rho_k)(v)=\rho_1(v)+\cdots+\rho_k(v)$, 其中 $\rho_i(v)\in\im\rho_i$.

        再证 $\im\rho_i\cap\left(\cup_{j\neq i}\im\rho_j\right)=\{0\}$: $\forall v\in\im\rho_i\cap\left(\cup_{j\neq i}\im\rho_j\right)\Longleftrightarrow v\in\im\rho_i\Longleftrightarrow\rho_i(v)=v$,\\
        且 $\exists j\neq i$, s.t. $v\in\im\rho_j\Longleftrightarrow\exists u\in V$, s.t. $\rho_j(u)=v$\\
        $\Longrightarrow v=\rho_i(v)=\rho_i(\rho_j(u))=\rho_i\circ\rho_j(u)=0\Longrightarrow\im\rho_i\cap\left(\sum_{j\neq i}\im\rho_j\right)=\{0\}$.

        综上, 得证.
        \item[(2)] $\forall v\in V$, $\because V=S_1\oplus\cdots\oplus S_k$, $\therefore v=v_1+\cdots+v_k$, 其中 $v_i=\rho_i(v)\in S_i$\\
        $\Longrightarrow v=\rho_1(v)+\cdots+\rho_k(v)=(\rho_1+\cdots+\rho_k)(v)\Longrightarrow\rho_1+\cdots+\rho_k=1$.\\
        $\forall v\in V$, $\rho_j(v)\in\im\rho_j=S_j\subseteq\sum_{j\neq i}S_i\Longrightarrow\rho_i\circ\rho_j(v)=\rho_i(\rho_j(v))=0$, 其中 $i\neq j\Longrightarrow\rho_i\circ\rho_j=0$.\\
        同理, $\rho_j\circ\rho_i=0$, 故 $\rho_i$ 与 $\rho_j$ 正交, 故 $\rho_1+\cdots\rho_k=1$ 为单位分解.
    \end{itemize}
\end{pf}

若 $V=\mathcal{E}_{\lambda_1}\oplus\cdots\oplus\mathcal{E}_{\lambda_k}$, 令 $\rho_i$ 为在 $\mathcal{E}_{\lambda_i}$ 上沿 $\sum_{j\neq i}\mathcal{E}_{\lambda_i}$ 的投影, 则 $\rho_1+\cdots+\rho_k=1$ 是单位分解,\\
$\rho_i|_{\mathcal{E}_{\lambda_i}}=1|_{\mathcal{E}_{\lambda_i}}\Longrightarrow\forall v\in\mathcal{E}_{\lambda_i}$, $\tau(v)=\lambda_iv=\lambda\rho_i(v)$, 即在 $\mathcal{E}_{\lambda_i}$ 上, $\tau|_{\mathcal{E}_{\lambda_i}}=\lambda\rho_i$, 从而引出定理 \ref{thm-8.10-8.11-8.18} (6).
\ifx\allfiles\undefined
\end{document}
\fi
