% !Tex program = pdflatex
% 第 13 章: 希尔伯特空间 Chap 13: Hilbert Spaces
\ifx\allfiles\undefined
\documentclass{note}
\setcounter{chapter}{+12}
\begin{document}
\fi
\chapter{希尔伯特空间}
$F=\mathbb{C}(\mathbb{R})$, $V$ 是 $F$ 上的内积向量空间, 则可诱导出范数 $\norm{v}=\sqrt{\langle v,v\rangle}$, 度量 $d(u,v)=\norm{u-v}=\sqrt{\langle u-v,u-v\rangle}$, 故 $(V,d)$ 为度量空间.\\
$v\in V$, $r>0$, $B(v,r)=\{u\in V\mid d(u,v)<r\}=\{u\in V\mid\sqrt{\langle u-v,u-v\rangle}<r\}$.\\
收敛: $(v_n)\rightarrow v\Longleftrightarrow\lim_{n\rightarrow\infty}d(v_n,v)=\lim_{n\rightarrow\infty}\sqrt{\langle v_n-v,v_n-v\rangle}=0$.\\
线性变换 $\tau:V\rightarrow W$ 连续 $\Longleftrightarrow(\tau(v_n))\rightarrow\tau(v)$.

\section{希尔伯特空间}
\begin{thm}[(课本定理 13.5)]
    \begin{itemize}
        \item[(1)] $(x_n)\rightarrow x$, $(y_n)\rightarrow y\Longrightarrow(\langle x_n,y_n\rangle)\rightarrow\langle x,y\rangle$.
        \item[(2)] 序列收敛 $\Longrightarrow$ 序列的范数收敛. $(x_n)\rightarrow x$, 则 $(\norm{x_n})\rightarrow\norm{x}$.
    \end{itemize}
\end{thm}
\begin{pf}
    \begin{itemize}
        \item[(2)] $(x_n)\rightarrow x\Longleftrightarrow\forall\epsilon>0$, $\exists N>0$, 当 $n>N$ 时, $d(x_n,x)=\norm{x_n-x}<\epsilon$\\
        $\Longrightarrow\abs{\norm{x_n}-\norm{x}}<\norm{x_n-x}<\epsilon$, 故 $(\norm{x_n})\rightarrow\norm{x}$.
        \item[(1)] $\abs{\langle x_n,y_n\rangle-\langle x,y\rangle}=\abs{\langle x_n,y_n\rangle-\langle x_n,y\rangle+\langle x_n,y\rangle-\langle x,y\rangle}\leq\abs{\langle x_n,y_n\rangle-\langle x_n,y\rangle}+\abs{\langle x_n,y\rangle-\langle x,y\rangle}=\abs{\langle x_n,y_n-y\rangle}+\abs{\langle x_n-x,y\rangle}\leq\norm{x_n}\norm{y_n-y}+\norm{x_n-x}\norm{y}$.\\
        $\because(x_n)\rightarrow x$, $(y_n)\rightarrow y$, $\therefore\norm{x_n-x}\rightarrow 0$, $\norm{y_n-y}\rightarrow 0\Longrightarrow\abs{\langle x_n,y_n\rangle-\langle x,y\rangle}\rightarrow 0\Longrightarrow(\langle x_n,y_n\rangle)\rightarrow\langle x,y\rangle$.
    \end{itemize}
\end{pf}

注意 (2) 反之不真, 如 $(x_n=(-1)^n)$ 的范数收敛, 但序列本身不收敛.

上述定理说明内积向量空间上天然 $\exists$ 连续映射 $\norm{}:V\rightarrow\mathbb{R}$, $v\mapsto\norm{v}$, $\langle x,\rangle:V\rightarrow F$, $v\mapsto\langle x,v\rangle$.

\begin{df}[希尔伯特空间]
    在内积诱导的度量下完备的度量空间.
\end{df}

\begin{thm}[完备化定理 (课本定理 13.6)]
    对内积向量空间 $V$, $\exists$ 希尔伯特空间 $H$ 及等距映射 $\tau:V\rightarrow H$, s.t. $\tau(V)$ 在 $H$ 中稠密.
\end{thm}

\begin{thm}[(课本定理 13.7)]
    $V$ 为内积向量空间, 子空间 $S\subseteq V$, 则
    \begin{itemize}
        \item[(1)] $S$ 完备 $\Longrightarrow S$ 闭.
        \item[(2)] $V$ 为希尔伯特空间, $S$ 为 $V$ 的子空间, $S$ 闭 $\Longleftrightarrow S$ 完备.
        \item[(3)] $\dim S<\infty$, 则 $S$ 闭且完备.
    \end{itemize}
\end{thm}
\begin{pf}
    \begin{itemize}
        \item[(1)(2)] 与定理 \ref{thm-12.6} 的证明同.
        \item[(3)] $\because\dim S=n$, $\therefore\exists$ 正交归一基 $\{b_1,\cdots,b_n\}$.\\
        取 $S$ 中序列 $(v_n)\rightarrow v\in V$, 下证 $v\in S$.\\
        假设 $v\notin S$, 则 $\hat{v}\in S$ 且 $v-\hat{v}\perp S$.\\
        $\norm{v_n-v}^2=\norm{v_n-\hat{v}+\hat{v}_n+\hat{v}}^2$, 其中 $v-\hat{v}\in S^{\perp}$, $\hat{v}_n+\hat{v}\in S$\\
        $\Longrightarrow\norm{v_n-v}^2=\norm{v-\hat{v}}^2-\norm{\hat{v}_n+v}^2\geq\norm{v-\hat{v}}^2>0$, 与 $(v_n)\rightarrow v$ 矛盾, 故假设错误, $v\in S\Longrightarrow S$ 闭.

        由完备化定理, $\exists$ 希尔伯特空间 $H$ 及等距映射 $\tau:S\rightarrow H$, s.t. $\tau(S)$ 在 $H$ 中稠密.\\
        $\because\tau$ 等距, $\therefore\tau$ 单射 $\Longrightarrow S$ 等距同构.\\
        又 $\because\dim S=n$, $\therefore\dim\tau(S)=n\Longrightarrow\tau(S)$ 闭.\\
        又 $\tau(S)$ 在 $H$ 中稠密, $\therefore H=\cl(\tau(S))=\tau(S)\Longrightarrow S\approx H$.\\
        又 $H$ 为希尔伯特空间, $\therefore S$ 为希尔伯特空间.
    \end{itemize}
\end{pf}

\section{无穷级数}
此前, 我们已经证明, 对给定的向量, 由有限维子空间有正交归一基可表示其在有限维子空间中的最佳近似; 由于无限维子空间有最大正交归一集, 但该最大正交归一集未必为基, 故向量在无限维子空间中未必有最佳近似; 以开子集为例可说明, 向量在子集中未必有最佳近似.

\begin{df}[级数收敛和绝对收敛]
    序列 $(x_n)$ 的前 $n$ 项和 $s_n=\sum_{i=1}^nx_i$, 若 $(s_n)$ 在 $V$ 中收敛于 $s$, 则称级数 $\sum_{i=1}^{\infty}x_i$ 在 $V$ 中\textbf{收敛}于 $s$, 记作 $\sum_{i=1}^{\infty}=s$; 若 $\sum_{i=1}^{\infty}\abs{x_i}$ 收敛, 则称 $\sum_{i=1}^{\infty}x_i$ 绝对收敛.
\end{df}

\begin{thm}[收敛和绝对收敛的关系 (课本定理 13.8)]
    内积向量空间 $V$ 完备 $\Longleftrightarrow V$ 上绝对收敛级数收敛.
\end{thm}
\begin{pf}
    ``$\Longrightarrow$'': 取绝对收敛级数 $\sum_{i=1}^{\infty}x_i$, 则 $\sum_{i=1}^{\infty}x_i$ 收敛.\\
    令 $s_n=\sum_{i=1}^n\norm{x_i}$, 则 $(s_n)$ 收敛 $\Longrightarrow(s_n)$ 柯西, 即 $\forall\epsilon>0$, $\exists N>0$, 当 $n,m>N$ 时, $\norm{s_n-s_m}<\epsilon$.\\
    无妨 $n<m$, 则 $\abs{s_n-s_m}=\abs{\sum_{i=1}^n\norm{x_i}-\sum_{i=1}^m\norm{x_i}}=\abs{\sum_{i=n+1}^m\norm{x_i}}<\epsilon$.\\
    令 $a_n=\sum_{i=1}^nx_i$, $\norm{a_n-a_m}=\norm{\sum_{i=1}^nx_i-\sum_{i=1}^mx_i}=\norm{\sum_{i=n+1}^mx_i}\leq\sum_{i=n+1}^m\norm{x_i}<\epsilon\Longrightarrow a_n=\sum_{i=1}^nx_i$ 柯西.\\
    又 $\because V$ 完备, $\therefore a_n$ 收敛 $\Longrightarrow\sum_{i=1}^{\infty}$ 在 $V$ 中收敛.

    ``$\Longleftarrow$'': 取 $V$ 中柯西序列 $(x_n)$, 则 $\forall\epsilon>0$, $\exists N>0$, s.t. 当 $n>N$ 时, $\norm{x_n-x_m}<\epsilon$.\\
    特别地, 令 $\epsilon_k=\frac{1}{2^k}$, 其中 $k=0,1,\cdots$, 则 $\forall\epsilon_k$, $\exists N_k>0$, s.t. 当 $n,m>N$ 时, $\norm{x_n-x_m}<\epsilon_k=\frac{1}{2^k}$.\\
    选 $N_1<N_2<\cdots$, 则 $\norm{x_{N_{k+1}}-x_{N_k}}<\frac{1}{2^k}\Longrightarrow\sum_{k=1}^{\infty}\norm{x_{N_{k+1}}-x_{N_k}}<\sum_{k=1}^{\infty}\frac{1}{2^k}=1\Longrightarrow\sum_{k=1}^{\infty}\norm{x_{N_{k+1}}-x_{N_k}}$ 收敛.\\
    又 $\because V$ 上绝对收敛级数收敛, $\therefore\sum_{k=1}^{\infty}x_{N_{k+1}}-x_{N_k}=x_{N_{n+1}}-x_{N_1}$ 收敛 $\Longrightarrow x_{N_{k+1}}$ 收敛.\\
    由引理 \ref{cor-13.1} 得 $x_n$ 收敛, 故 $V$ 完备.

    综上, 得证.
\end{pf}

\begin{cor}\label{cor-13.1}
    柯西序列的子列收敛, 则其必收敛.
\end{cor}
\begin{pf}
    设柯西序列 $(x_n)$ 的子列 $(x_{N_k})\rightarrow x$, 即 $\forall\epsilon>0$, $\exists K>0$, s.t. 当 $k>K$ 时, $\norm{x_{N_k}-x}<\frac{\epsilon}{2}$.\\
    $\because(x_n)$ 柯西, $\therefore\forall\epsilon$, $\exists N>0$, s.t. 当 $n,m>N$ 时, $\norm{x_n-x_m}<\frac{\epsilon}{2}$.\\
    取 $m=\max\{N+1,N_K+1\}$, $\norm{x_n-x}=\norm{x_n-x_m+x_m-x}\leq\norm{x_n-x_m}+\norm{x_m-x}<\epsilon\Longrightarrow(x_n)\rightarrow x$.
\end{pf}

\section{近似问题}
\begin{df}[凸集]
    若 $\forall x,y\in C$, $\forall p\in[0,1]$, $px+(1-p)y\in C$, 则称 $C$ 为凸集.
\end{df}
\begin{eg}
    三角形、矩形、圆形均为凸集.\\
    三角形中任一点可表为其三个顶点的凸组合 ($\sum_ip_ix_i$, 其中 $p_i>0\forall i$, $\sum_ip_i=1$).\\
    无法写成其他点的凸组合的点称为\textbf{极点}, 如三角形的三个顶点.\\
    过凸集的极限点且将整个空间划分为含有凸集和不含凸集的两部分的超平面称为面 (face), 如过三角形的一个顶点但未过三角形内部的直线.\\
    过凸集的不止一个极限点的面, 称为 facet, 如三角形的边所在的直线.
\end{eg}

\begin{thm}[(课本定理 13.9)]
    $S$ 为内积向量空间 $V$ 的完备的凸闭子集, 则 $\forall x\in V$, $\exists!\hat{x}\in S$, s.t. $\norm{x-\hat{x}}=\inf_{y\in S}\norm{x-y}$, 称 $\hat{x}$ 为 $x$ 在 $S$ 中的最佳近似.
\end{thm}
\ifx\allfiles\undefined
\end{document}
\fi