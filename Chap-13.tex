% !Tex program = pdflatex
% 第 13 章: 希尔伯特空间 Chap 13: Hilbert Spaces
\ifx\allfiles\undefined
\documentclass{note}
\setcounter{chapter}{+12}
\begin{document}
\fi
\chapter{希尔伯特空间}
$F=\mathbb{C}(\mathbb{R})$, $V$ 是 $F$ 上的内积向量空间, 则可诱导出范数 $\norm{v}=\sqrt{\langle v,v\rangle}$, 度量 $d(u,v)=\norm{u-v}=\sqrt{\langle u-v,u-v\rangle}$, 故 $(V,d)$ 为度量空间.\\
$v\in V$, $r>0$, $B(v,r)=\{u\in V\mid d(u,v)<r\}=\{u\in V\mid\sqrt{\langle u-v,u-v\rangle}<r\}$.\\
收敛: $(v_n)\rightarrow v\Longleftrightarrow\lim_{n\rightarrow\infty}d(v_n,v)=\lim_{n\rightarrow\infty}\sqrt{\langle v_n-v,v_n-v\rangle}=0$.\\
线性变换 $\tau:V\rightarrow W$ 连续 $\Longleftrightarrow(\tau(v_n))\rightarrow\tau(v)$.

\section{希尔伯特空间}
\begin{thm}[(课本定理 13.5)]
    \begin{itemize}
        \item[(1)] $(x_n)\rightarrow x$, $(y_n)\rightarrow y\Longrightarrow(\langle x_n,y_n\rangle)\rightarrow\langle x,y\rangle$.
        \item[(2)] 序列收敛 $\Longrightarrow$ 序列的范数收敛. $(x_n)\rightarrow x$, 则 $(\norm{x_n})\rightarrow\norm{x}$.
    \end{itemize}
\end{thm}
\begin{pf}
    \begin{itemize}
        \item[(2)] $(x_n)\rightarrow x\Longleftrightarrow\forall\epsilon>0$, $\exists N>0$, 当 $n>N$ 时, $d(x_n,x)=\norm{x_n-x}<\epsilon$\\
        $\Longrightarrow\abs{\norm{x_n}-\norm{x}}<\norm{x_n-x}<\epsilon$, 故 $(\norm{x_n})\rightarrow\norm{x}$.
        \item[(1)] $\abs{\langle x_n,y_n\rangle-\langle x,y\rangle}=\abs{\langle x_n,y_n\rangle-\langle x_n,y\rangle+\langle x_n,y\rangle-\langle x,y\rangle}\leq\abs{\langle x_n,y_n\rangle-\langle x_n,y\rangle}+\abs{\langle x_n,y\rangle-\langle x,y\rangle}=\abs{\langle x_n,y_n-y\rangle}+\abs{\langle x_n-x,y\rangle}\leq\norm{x_n}\norm{y_n-y}+\norm{x_n-x}\norm{y}$.\\
        $\because(x_n)\rightarrow x$, $(y_n)\rightarrow y$, $\therefore\norm{x_n-x}\rightarrow 0$, $\norm{y_n-y}\rightarrow 0\Longrightarrow\abs{\langle x_n,y_n\rangle-\langle x,y\rangle}\rightarrow 0\Longrightarrow(\langle x_n,y_n\rangle)\rightarrow\langle x,y\rangle$.
    \end{itemize}
\end{pf}

注意 (2) 反之不真, 如 $(x_n=(-1)^n)$ 的范数收敛, 但序列本身不收敛.

上述定理说明内积向量空间上天然 $\exists$ 连续映射 $\norm{}:V\rightarrow\mathbb{R}$, $v\mapsto\norm{v}$, $\langle x,\rangle:V\rightarrow F$, $v\mapsto\langle x,v\rangle$.

\begin{df}[希尔伯特空间]
    在内积诱导的度量下完备的度量空间.
\end{df}

\begin{thm}[完备化定理 (课本定理 13.6)]
    对内积向量空间 $V$, $\exists$ 希尔伯特空间 $H$ 及等距映射 $\tau:V\rightarrow H$, s.t. $\tau(V)$ 在 $H$ 中稠密.
\end{thm}

\begin{thm}[(课本定理 13.7)]
    $V$ 为内积向量空间, 子空间 $S\subseteq V$, 则
    \begin{itemize}
        \item[(1)] $S$ 完备 $\Longrightarrow S$ 闭.
        \item[(2)] $V$ 为希尔伯特空间, $S$ 为 $V$ 的子空间, $S$ 闭 $\Longleftrightarrow S$ 完备.
        \item[(3)] $\dim S<\infty$, 则 $S$ 闭且完备.
    \end{itemize}
\end{thm}
\begin{pf}
    \begin{itemize}
        \item[(1)(2)] 与定理 \ref{thm-12.6} 的证明同.
        \item[(3)] $\because\dim S=n$, $\therefore\exists$ 正交归一基 $\{b_1,\cdots,b_n\}$.\\
        取 $S$ 中序列 $(v_n)\rightarrow v\in V$, 下证 $v\in S$.\\
        假设 $v\notin S$, 则 $\hat{v}\in S$ 且 $v-\hat{v}\perp S$.\\
        $\norm{v_n-v}^2=\norm{v_n-\hat{v}+\hat{v}_n+\hat{v}}^2$, 其中 $v-\hat{v}\in S^{\perp}$, $\hat{v}_n+\hat{v}\in S$\\
        $\Longrightarrow\norm{v_n-v}^2=\norm{v-\hat{v}}^2-\norm{\hat{v}_n+v}^2\geq\norm{v-\hat{v}}^2>0$, 与 $(v_n)\rightarrow v$ 矛盾, 故假设错误, $v\in S\Longrightarrow S$ 闭.

        由完备化定理, $\exists$ 希尔伯特空间 $H$ 及等距映射 $\tau:S\rightarrow H$, s.t. $\tau(S)$ 在 $H$ 中稠密.\\
        $\because\tau$ 等距, $\therefore\tau$ 单射 $\Longrightarrow S$ 等距同构.\\
        又 $\because\dim S=n$, $\therefore\dim\tau(S)=n\Longrightarrow\tau(S)$ 闭.\\
        又 $\tau(S)$ 在 $H$ 中稠密, $\therefore H=\cl(\tau(S))=\tau(S)\Longrightarrow S\approx H$.\\
        又 $H$ 为希尔伯特空间, $\therefore S$ 为希尔伯特空间.
    \end{itemize}
\end{pf}

\section{无穷级数}
此前, 我们已经证明, 对给定的向量, 由有限维子空间有正交归一基可表示其在有限维子空间中的最佳近似; 由于无限维子空间有最大正交归一集, 但该最大正交归一集未必为基, 故向量在无限维子空间中未必有最佳近似; 以开子集为例可说明, 向量在子集中未必有最佳近似.

\begin{df}[级数收敛和绝对收敛]
    序列 $(x_n)$ 的前 $n$ 项和 $s_n=\sum_{i=1}^nx_i$, 若 $(s_n)$ 在 $V$ 中收敛于 $s$, 则称级数 $\sum_{i=1}^{\infty}x_i$ 在 $V$ 中\textbf{收敛}于 $s$, 记作 $\sum_{i=1}^{\infty}=s$; 若 $\sum_{i=1}^{\infty}\abs{x_i}$ 收敛, 则称 $\sum_{i=1}^{\infty}x_i$ 绝对收敛.
\end{df}

\begin{thm}[收敛和绝对收敛的关系 (课本定理 13.8)]
    内积向量空间 $V$ 完备 $\Longleftrightarrow V$ 上绝对收敛级数收敛.
\end{thm}
\begin{pf}
    ``$\Longrightarrow$'': 取绝对收敛级数 $\sum_{i=1}^{\infty}x_i$, 则 $\sum_{i=1}^{\infty}x_i$ 收敛.\\
    令 $s_n=\sum_{i=1}^n\norm{x_i}$, 则 $(s_n)$ 收敛 $\Longrightarrow(s_n)$ 柯西, 即 $\forall\epsilon>0$, $\exists N>0$, 当 $n,m>N$ 时, $\norm{s_n-s_m}<\epsilon$.\\
    无妨 $n<m$, 则 $\abs{s_n-s_m}=\abs{\sum_{i=1}^n\norm{x_i}-\sum_{i=1}^m\norm{x_i}}=\abs{\sum_{i=n+1}^m\norm{x_i}}<\epsilon$.\\
    令 $a_n=\sum_{i=1}^nx_i$, $\norm{a_n-a_m}=\norm{\sum_{i=1}^nx_i-\sum_{i=1}^mx_i}=\norm{\sum_{i=n+1}^mx_i}\leq\sum_{i=n+1}^m\norm{x_i}<\epsilon\Longrightarrow a_n=\sum_{i=1}^nx_i$ 柯西.\\
    又 $\because V$ 完备, $\therefore a_n$ 收敛 $\Longrightarrow\sum_{i=1}^{\infty}$ 在 $V$ 中收敛.

    ``$\Longleftarrow$'': 取 $V$ 中柯西序列 $(x_n)$, 则 $\forall\epsilon>0$, $\exists N>0$, s.t. 当 $n>N$ 时, $\norm{x_n-x_m}<\epsilon$.\\
    特别地, 令 $\epsilon_k=\frac{1}{2^k}$, 其中 $k=0,1,\cdots$, 则 $\forall\epsilon_k$, $\exists N_k>0$, s.t. 当 $n,m>N$ 时, $\norm{x_n-x_m}<\epsilon_k=\frac{1}{2^k}$.\\
    选 $N_1<N_2<\cdots$, 则 $\norm{x_{N_{k+1}}-x_{N_k}}<\frac{1}{2^k}\Longrightarrow\sum_{k=1}^{\infty}\norm{x_{N_{k+1}}-x_{N_k}}<\sum_{k=1}^{\infty}\frac{1}{2^k}=1\Longrightarrow\sum_{k=1}^{\infty}\norm{x_{N_{k+1}}-x_{N_k}}$ 收敛.\\
    又 $\because V$ 上绝对收敛级数收敛, $\therefore\sum_{k=1}^{\infty}x_{N_{k+1}}-x_{N_k}=x_{N_{n+1}}-x_{N_1}$ 收敛 $\Longrightarrow x_{N_{k+1}}$ 收敛.\\
    由引理 \ref{cor-13.1} 得 $x_n$ 收敛, 故 $V$ 完备.

    综上, 得证.
\end{pf}

\begin{cor}\label{cor-13.1}
    柯西序列的子列收敛, 则其必收敛.
\end{cor}
\begin{pf}
    设柯西序列 $(x_n)$ 的子列 $(x_{N_k})\rightarrow x$, 即 $\forall\epsilon>0$, $\exists K>0$, s.t. 当 $k>K$ 时, $\norm{x_{N_k}-x}<\frac{\epsilon}{2}$.\\
    $\because(x_n)$ 柯西, $\therefore\forall\epsilon$, $\exists N>0$, s.t. 当 $n,m>N$ 时, $\norm{x_n-x_m}<\frac{\epsilon}{2}$.\\
    取 $m=\max\{N+1,N_K+1\}$, $\norm{x_n-x}=\norm{x_n-x_m+x_m-x}\leq\norm{x_n-x_m}+\norm{x_m-x}<\epsilon\Longrightarrow(x_n)\rightarrow x$.
\end{pf}

\section{近似问题}
\begin{df}[凸集]
    若 $\forall x,y\in C$, $\forall p\in[0,1]$, $px+(1-p)y\in C$, 则称 $C$ 为凸集.
\end{df}
\begin{eg}
    三角形、矩形、圆形均为凸集.\\
    三角形中任一点可表为其三个顶点的\textbf{凸组合} ($\sum_ip_ix_i$, 其中 $p_i>0\forall i$, $\sum_ip_i=1$).\\
    无法写成其他点的凸组合的点称为\textbf{极点}, 如三角形的三个顶点.\\
    过凸集的极限点且将整个空间划分为含有凸集和不含凸集的两部分的超平面称为\textbf{面 (face)}, 如过三角形的一个顶点但未过三角形内部的直线.\\
    过凸集的不止一个极限点的面, 称为 \textbf{facet}, 如三角形的边所在的直线.
\end{eg}

\begin{thm}[(课本定理 13.9)]\label{thm-13.9}
    $V$ 为内积向量空间, $S$ 为 $V$ 的完备的凸闭子集, 则 $\forall x\in V$, $\exists!\hat{x}\in S$, s.t. $\norm{x-\hat{x}}=\inf_{y\in S}\norm{x-y}$, 称 $\hat{x}$ 为 $x$ 在 $S$ 中的最佳近似.
\end{thm}
\begin{pf}
    令 $\delta=\inf_{y\in S}\norm{x-y}$, 则 $\forall\epsilon>0$, $\exists z\in S$, s.t. $\norm{y-z}<\delta+\epsilon$.\\
    取 $\epsilon=\frac{1}{n}$, $\exists y_1\in S$, s.t. $\norm{x-y_1}<\delta+\epsilon_1$,\\
    $\cdots$,\\
    $\exists y_n\in S$, s.t. $\norm{x-y_n}\leq\delta+\epsilon_n$,\\
    $\cdots$, 于是得序列 $(y_n)\in S$.\\
    令 $s_n=x-y_n$, 则 $(\norm{s_n})\rightarrow\delta$.\\
    $\norm{s_i-s_j}^2=2(\norm{s_i}^2+\norm{s_j})-\norm{s_i+s_j}^2$, 其中 $\norm{s_i+s_j}^2=\norm{x-y_i+x-y_j}^2=4\norm{x-\frac{y_i+y_j}{2}}^2$,\\
    $\because y_i,y_j\in S$ 且 $S$ 凸, $\therefore\frac{y_i+y_j}{2}\in S\Longrightarrow\norm{x-\frac{y_i+y_j}{2}}^2\geq\delta$\\
    $\Longrightarrow\norm{s_i-s_j}^2\leq 2(\norm{s_i}^2+\norm{s_j}^2)-4\delta^2\rightarrow 4\delta^2-4\delta^2=0\Longrightarrow\norm{s_i-s_j}^2=\norm{x-y_i-x+y_j}^2=\norm{y_i-y_j}^2\rightarrow 0$, 故 $(y_n)$ 为 $S$ 中柯西列.\\
    又 $\because S$ 完备, $\therefore(y_n)\rightarrow x\Longrightarrow\norm{x-\hat{x}}=\lim_{n\rightarrow\infty}\norm{x-y_n}=\delta$.

    下证 $\hat{x}$ 的唯一性: 设 $w\in S$, s.t. $\norm{x-w}=\delta$.\\
    $\norm{\hat{x}-w}^2=\norm{\hat{x}-x+x-w}^2=2(\norm{x-\hat{x}}^2+\norm{x-w}^2)-\norm{\hat{x}-x-(x-w)}^2$, 其中 $\norm{\hat{x}-x-(x-w)}=4\norm{x-\frac{\hat{x}+w}{2}}^2$,\\
    $\because\hat{x},w\in S$ 且 $S$ 凸, $\therefore\frac{\hat{x}+w}{2}\in S\Longrightarrow\norm{x-\frac{\hat{x}+w}{2}}^2\geq\delta$\\
    $\Longrightarrow\norm{\hat{x}-w}^2\leq 2(\norm{x-\hat{x}}^2+\norm{x-w}^2)-4\delta^2=4\delta^2-4\delta^2=0\Longrightarrow\norm{\hat{x}-w}^2=0\Longrightarrow w=\hat{x}$.

    综上, 得证.
\end{pf}

若 $S$ 非凸, 则 $\hat{x}$ 未必唯一.

由上述定理, 可定义 $\min_{y\in S}\norm{x-y}\equiv\inf_{y\in S}\norm{x-y}$.

\begin{thm}[(课本定理 13.10)]
    $V$ 为内积向量空间, $S$ 为 $V$ 的完备子空间, 则 $\forall x\in V$, $\exists x$ 在 $S$ 中的最佳近似 $\hat{x}$ 且 $x-\hat{x}\perp S$.
\end{thm}
\begin{pf}
    $S$ 为子空间 $\Longleftrightarrow S$ 中线性运算封闭 $\Longrightarrow S$ 凸, 由定理 \ref{thm-13.9} 得, $\exists!x$ 在 $S$ 中的最佳近似 $\hat{x}\in S$, s.t. $\norm{x-\hat{x}}=\inf_{y\in S}\norm{x-y}$.

    下证 $x-\hat{x}\in S$: $\forall y\in S$, $r\in F$, $\norm{v-ry}^2=\langle v-ry,v-ry\rangle=\langle v,v\rangle+\langle v,-ry\rangle+\langle-ry,v\rangle+\langle-ry,-ry\rangle=\norm{v}^2-\bar{r}\langle v,y\rangle-r\langle y,v\rangle+\abs{r}^2\norm{y}^2=\norm{v}^2+\norm{y}^2\left(r\bar{r}-\bar{r}\frac{\langle v,y\rangle}{\norm{y}^2}-r\frac{\langle y,v\rangle}{\norm{y}^2}\right)=\norm{v}^2+\norm{y}^2\left(r-\frac{\langle v,y\rangle}{\norm{y}^2}\right)\left(\bar{r}-\frac{\langle y,v\rangle}{\norm{y}^2}\right)-\norm{y}^2\frac{\langle v,y\rangle\langle y,v\rangle}{\norm{y}^4}=\norm{x}^2+\norm{y}^2\abs{r-\frac{\langle v,y\rangle}{\norm{y}^2}}-\frac{\abs{\langle v,y\rangle}}{\norm{y}^2}$.\\
    当 $r=\frac{\langle v,y\rangle}{\norm{y}^2}$ 时, $\norm{v-ry}^2$ 取最小值, 此时 $\norm{v-ry}^2=\norm{v}^2-\frac{\abs{\langle v,y\rangle}^2}{\norm{y}^2}$.\\
    特别地, 令 $v=x-\hat{x}$, 则 $\norm{x-\hat{x}-ry}^2\geq\norm{x-\hat{x}}^2-\frac{\abs{\langle x-\hat{x},y\rangle}^2}{\norm{y}^2}$.\\
    又 $\because\norm{x-\hat{x}}=\inf_{y\in S}\norm{x-y}^2\leq\norm{x-(\hat{x}+ry)}^2$, $\therefore\frac{\abs{\langle x-\hat{x},y\rangle}^2}{\norm{y}^2}=0\Longrightarrow x-\hat{x}\perp y\Longrightarrow x-\hat{x}\perp S$.

    综上, 得证.
\end{pf}

\begin{thm}[投影定理 (课本定理 13.11)]
    $V$ 为内积向量空间, $S$ 为 $V$ 的完备子集, 则 $V=S\odot S^{\perp}$.\\
    特别地, 若 $H$ 为希尔伯特空间, $S$ 为 $H$ 的闭子空间, 则 $H=S\odot S^{\perp}$.
\end{thm}

\begin{thm}[(课本定理 13.12)]
    $S,T$ 为 $V$ 的子空间, 则
    \begin{itemize}
        \item[(1)] 若 $V=S\odot T$, 则 $V=S\odot T$.
        \item[(2)] 若正交补 $\exists$, 则唯一. 若 $S\odot T=S\odot T'$, 则 $T=T'$.
    \end{itemize}
\end{thm}
\begin{pf}
    \begin{itemize}
        \item[(1)] $\because V=S\odot T$, $\therefore\forall x\in T$, $x\perp S\Longrightarrow T\subseteq S^{\perp}$.

        $\forall w\in S^{\perp}\subseteq V=S\odot T$, $w=w_S+w_T$, 其中 $w_S\in S$, $w_T\in T$.\\
        $\because w\in S^{\perp}$, $w_S\in S$, $\therefore 0=\langle w_S,w\rangle=\langle w_S,w_S+w_T\rangle=\langle w_S,w_S\rangle+\msout{\langle w_S,w_T\rangle}0\Longrightarrow w_S=0\Longrightarrow w=w_T\in T\Longrightarrow S^{\perp}\subseteq T$.

        综上, 得证.
    \end{itemize}
\end{pf}

补空间不唯一, 但补空间之间互相同构.

\begin{thm}[(课本定理 13.13)]
    $H$ 为希尔伯特空间, 则
    \begin{itemize}
        \item[(1)] 若 $A\subseteq H$, 则 $\cspan(A)^{\perp}=A^{\perp\perp}~$\footnote{$\cspan(A)=\cl(\spanned{A})$ 为 $A$ 中向量所张成的空间的闭包.}.
        \item[(2)] 若 $S$ 为 $H$ 的子空间, 则 $\cl(S)=S^{\perp}$.
        \item[(3)] 若 $K$ 为 $H$ 的完备子空间, 则 $K=K^{\perp\perp}$.
    \end{itemize}
\end{thm}
\begin{pf}
    \begin{itemize}
        \item[(1)] 即证 $\cspan(A)^{\perp}=A^{\perp}$: $\because A\subseteq\cspan(A)$, $\therefore\cspan(A)^{\perp}\subseteq A^{\perp}$.

        $\cspan(A)=\spanned(A)\cup\lp(\spanned(A))$.\\
        $\forall x\in A$, $x\perp A$.\\
        $\forall y\in\spanned(A)$, $y=\sum_ir_ix_i$, 其中 $r_i\in F$, $x_i\in A$.\\
        $\because x\perp A$, $\therefore\langle x,x_i\rangle=0\forall i\Longrightarrow\langle x,y\rangle=\langle x,\sum_ir_ix_i\rangle=\sum_i\overline{r_i}\langle x,x_i\rangle=0\Longrightarrow x\perp\spanned(A)$.\\
        $\forall z\in\lp(\spanned(A))$, $\forall r$, $B(z,r)\cap\spanned(A)$ 中包含异于 $z$ 的点.\\
        取 $r_n=\frac{1}{n}$, 则可得序列 $(z_n)\in\spanned(A)$. s.t. $z_n\neq z\forall n$ 且 $B(z_n,z)<r_n$ 即 $(z_n)\rightarrow z$.\\
        $\because x\in A^{\perp}$, $\therefore x\perp\spanned(A)$.\\
        又 $\because(z_n)\in\spanned(A)$, $\therefore x\perp z_n$ 即 $\langle x,z_n\rangle=0\forall n$.\\
        又 $\because\langle x,\rangle:V\rightarrow\mathbb{R}$, $v\mapsto\langle x,v\rangle$ 连续, $(z_n)\rightarrow z$, $\therefore\langle x,z_n\rangle\rightarrow\langle x,z\rangle$.\\
        又 $\because\langle x,z_n\rangle=0$ 为常序列, $\therefore\langle x,z\rangle=0\Longrightarrow x\perp\lp(\spanned(A))$.\\
        $\because x\perp\spanned(A)$, $x\perp\lp(\spanned(A))$, $\therefore x\perp\spanned(A)\cup\lp(\spanned(A))=\cspan(A)\Longrightarrow A^{\perp}\subseteq\cspan(A)$.

        综上, $\cspan(A)=A^{\perp\perp}\Longrightarrow H=\cspan(A)\odot\cspan(A)^{\perp}=\cspan(A)\odot A^{\perp}\Longrightarrow\cspan(A)=A^{\perp\perp}$.
        \item[(2)] 将 (1) 中的 $\spanned(A)$ 换成 $S$ 即得证.
        \item[(3)] $\because K$ 为 $H$ 的完备子空间, $\therefore K$ 闭 $\Longrightarrow K=\cl(K)$, 利用 (2) 即得证.
    \end{itemize}
\end{pf}

利用上述定理可得如下推论:
\begin{thm}[(课本定理 13.14)]
    $H$ 为希尔伯特空间, $A\subseteq H$, 若 $\spanned(A)$ 在 $H$ 中稠密, 则 $A^{\perp}=\{0\}$.
\end{thm}
\begin{pf}
    $\because\spanned(A)$ 在 $H$ 中稠密, $\therefore\cspan(A)=\cl(\spanned(A))=H$.\\
    $\because H$ 为希尔伯特空间, $A\subseteq H$, $\therefore\cspan(A)=A^{\perp\perp}\Longrightarrow H=A^{\perp\perp}\Longrightarrow A^{\perp}=H^{\perp}=\{0\}$.\\
\end{pf}

\begin{thm}[(课本定理 13.15)]
    $\mathcal{O}$ 为 Hilbert 基, 则 $\mathcal{O}^{\perp}=\{0\}$.
\end{thm}
\begin{pf}
    $\because\mathcal{O}$ 为 Hilbert 基, $\therefore\mathcal{O}$ 为极大的正交集, 即得证.
\end{pf}

\begin{thm}[(课本定理 13.25)]
    $H$ 为 Hilbert 空间, $\mathcal{O}=\{o_k\mid k\in K\}$ 为 $H$ 中的正交归一簇, 则 $\forall x\in H$, $\hat{x}=\sum_{k\in K}\langle x,o_k\rangle$ 在 $H$ 中收敛, 为 $x$ 在 $\spanned(\mathcal{O})$ 中的最佳近似, $\sum_{k\in K}\abs{\langle x,o_k\rangle}^2\leq\norm{x}^2$.
\end{thm}

\begin{thm}[(课本定理 13.26)]
    $H$ 为 Hilbert 空间, $\mathcal{O}=\{u_k\mid k\in K\}$ 为 $H$ 中的正交归一簇, 则下列叙述等价:
    \begin{itemize}
        \item[(1)] $\mathcal{O}$ 为 Hilbert 基.
        \item[(2)] $\mathcal{O}=\{0\}$.
        \item[(3)] $x=\hat{x}$.
        \item[(4)] $\norm{x}=\norm{\hat{x}}$.
    \end{itemize}
\end{thm}
\ifx\allfiles\undefined
\end{document}
\fi