% !Tex program = pdflatex
% 第 1 章: 向量空间 Chap 1: Vector Space
\ifx\allfiles\undefined
\documentclass{note}
\setcounter{chapter}{0}
\begin{document}
\fi
\chapter{向量空间}
\begin{df}[向量空间]
    交换群 $(V,+)$ 和域 $F$, 数乘映射 $\alpha:F\times V\rightarrow V$, 若满足
    \begin{itemize}
        \item[(1)] $\alpha(r,u+v)=\alpha(r,u)+\alpha(r,v)$ (可简写为 $r(u+v)=ru+rv$)
        \item[(2)] $\alpha(r+t,u)=\alpha(r,u)+\alpha(t,u)$ (可简写为 $(r+t)u=ru+tu$)
        \item[(3)] $\alpha(r\cdot t,u)=\alpha(r,\alpha(t,u))$ (可简写为 $(rt)u=r(tu)$)
        \item[(4)] \textbf{有单位元}: $\exists 1\in F$, s.t. $\alpha(1,u)=u$ (可简写为 $1u=u$)
    \end{itemize}
    则称 $V$ 是 $F$ 上的\textbf{向量空间}.
\end{df}

\begin{eg}[直角坐标系]
    $(\mathbb{R},+,\cdot)$ 为域, $(\mathbb{R}^2\equiv\{(x,y)\mid x,y\in\mathbb{R}\},+)$ 为交换群, 满足
    \begin{itemize}
        \item[(1)] $r((x_1,y_1)+(x_2,y_2))=r(x_1+x_2,y_1+y_2)=(rx_1+rx_2,ry_1+ry_2)=(rx_1,ry_1)+(rx_2,ry_2)=r(x_1,y_1)+r(x_2,y_2)$
        \item[(2)] $(r+t)(x,y)=((r+t)x,(r+t)y)=(rx+tx,ry+ty)=(rx,ry)+(tx,ty)=r(x,y)+t(x,y)$
        \item[(3)] $(r\cdot t)(x,y)=(rtx,rty)=r(tx,ty)=r(t(x,y))$
        \item[(4)] $1(x,y)=(x,y)$
    \end{itemize}
    故 $\mathbb{R}^2$ 为 $\mathbb{R}$ 上的向量空间.
\end{eg}

$0v=0$. (注意两个 $0$ 的区别, 等号左边的 $0$ 为 域 $F$ 中的零元, 等号右边的 $0$ 为 $V$ 中的零向量.)
\begin{pf}
    $0v=(0+0)v=0v+0v\Longrightarrow 0v=0$.
\end{pf}

$r\in F$, $0\in V$, 则 $r0=0$.
\begin{pf}
    $r0=r(0+0)=r0+r0\Longrightarrow r0=0$.
\end{pf}

$-1v=-v$.
\begin{pf}
    $-1v=-(1v)=-v$.
\end{pf}

\begin{eg}
    $\mathbb{R}^2$ 为 $\mathbb{R}$ 上的向量空间.\\
    $\mathbb{R}^2$ 为 $\mathbb{Q}$ 上的向量空间.\\
    $\because$ 对 $c\in\mathbb{C}$, $v\in\mathbb{R}^2$, $cv\notin\mathbb{R}^2$, $\therefore\mathbb{R}^2$ 不是 $\mathbb{C}$ 上的向量空间.
\end{eg}

\begin{eg}
    $F^n\equiv\{(r_1,\cdots,r_n)\mid r_i\in F\}$, 满足 $(r_1,\cdots,r_n)+(l_1,\cdots,l_n)=(r_1+l_1,\cdots,r_n+l_n)$, $r(r_1,\cdots,r_n)=(rr_1,\cdots,rr_n)$. $F^n$ 为 $F$ 上的向量空间.
\end{eg}
\begin{pf}
    $\because r((r_1,\cdots,r_n)+(l_1,\cdots,l_n))=r(r_1+l_1,\cdots,r_n+l_n)=(rr_1+rl_1,\cdots,rr_n+rl_n)=(rr_1,\cdots,rr_n)+(rl_1,\cdots,rl_n)=r(r_1,\cdots,r_n)+r(l_1,\cdots,l_n)$,\\
    且 $(r+t)(r_1,\cdots,r_n)=((r+t)r_1,\cdots,(r+t)r_n)=(rr_1+tr_1,\cdots,rr_n+tr_n)=(rr_1,\cdots,rr_n)+(tr_1,\cdots,tr_n)=r(r_1,\cdots,r_n)+t(r_1,\cdots,r_n)$,\\
    且 $(r\cdot t)(r_1,\cdots,r_n)=(rtr_1,\cdots,rtr_n)=r(tr_1,\cdots,tr_n)=r(t(r_1,\cdots,r_n))$,\\
    且 $1(r_1,\cdots,r_n)=(r_1,\cdots,r_n)$,\\
    $\therefore F^n$ 为 $F$ 上的向量空间.
\end{pf}

\begin{df}[子空间]
    $\emptyset\neq S\subseteq V$, 若 $S$ 为 $V$ 的子群, 且在相同的数乘下构成 $F$ 上的向量空间, 则称 $S$ 是 $V$ 的\textbf{子空间}.
\end{df}

\begin{thm}[子空间的判定 (课本定理 1.1)]
    $S$ 为 $V$ 的子空间 $\Longleftrightarrow\forall a,b\in S$, $r,t\in F$, $ra+tb\in S$ (即线性运算封闭).
\end{thm}
\begin{pf}
    ``$\Longrightarrow$'': $ra\in S$, $-tb\in S$, 又 $\because S$ 为 $V$ 的子群, $ra-(-tb)\in S$.

    ``$\Longrightarrow$'': 令 $r=1$, $t=-1$, 有 $a-b\in S\Longrightarrow S<V$.\\
    令 $t=0$, 有 $ra\in S$, 故 $S$ 为 $V$ 的子空间.

    综上, 得证.
\end{pf}

子空间的交是子空间.
\begin{pf}
    设 $S_1,\cdots,S_n$ 为 $V$ 的子空间, 则 $S_1,\cdots,S_n$ 为 $V$ 的子群 $\Longrightarrow\cap_{i=1}^nS_i$ 为 $V$ 的子群.

    $\forall u,v\in\cap_{i=1}^nS_i$, $\forall k$, $u,v\in S_k\Longrightarrow u,v$ 满足与 $F$ 中向量相同的数乘映射.

    综上, 得证.
\end{pf}

$S,T$ 是 $V$ 的子空间, $S+V\equiv\{u+v\mid u\in S,v\in T\}$ 为 $V$ 的子空间.
\begin{pf}
    $\forall w_1,w_2\in S+T$, $r,t\in F$,\\
    $w_1\in S+T\Longrightarrow w_1=u_1+v_1$, $u_1\in S$, $v_1\in T$,\\
    $w_2\in S+T\Longrightarrow w_2=u_2+v_2$, $u_2\in S$, $v_2\in T$.\\
    $rw_1+tw_2=r(u_1+v_1)+t(u_2+v_2)=(ru_1+tu_2)+(rv1+tv_2)$, 其中 $ru_1+tu_2\in S$, $rv_1+tv_2\in T\Longrightarrow rw_1+tw_2\in S+T$, 故 $S+T$ 为 $V$ 的子空间.
\end{pf}

\begin{df}[生成子空间和生成集]
    $\emptyset\neq S\subseteq V$, $S$ 的\textbf{生成子空间}为 $\langle S\rangle\equiv$ 包含 $S$ 的最小子空间 $=\left\{\sum_{i=1}^nr_iu_i\mid r_i\in F,u_i\in S,n\in\mathbb{N}\right\}$, 其中称 $S$ 为\textbf{生成集}.
\end{df}

\begin{eg}
    向量空间 $\mathbb{R}^2$,\\
    $S_x=\langle\{(1,0)\}\rangle=\{(x,0)\mid x\in\mathbb{R}\}=x$ 轴,\\
    $S_y=\langle\{(0,1)\}\rangle=\{(0,y)\mid y\in\mathbb{R}\}=y$ 轴,\\
    $\langle\{(1,0),(0,1)\}\rangle=\langle\{(1,1),(1,-1)\}\rangle=\mathbb{R}^2$, 故对同一生成子空间, 生成集不唯一.
\end{eg}

\begin{df}[线性无关]
    非零元 $u_1,\cdots,u_m$, 若 $r_1u_1+\cdots+\cdots+r_mu_m=0\Longrightarrow r_1=\cdots=r_m=0$, 则称 $u_1,\cdots,u_m$ \textbf{线性无关}.\\
    若 $S$ 中任意有限个元素线性无关, 则称 $S$ \textbf{线性无关}.
\end{df}

\begin{eg}
    $(1,0)$ 与 $(0,1)$ 线性无关.
\end{eg}
\begin{pf}
    $r_1(1,0)+r_2(0,1)=(r_1,r_2)=0=(0,0)\Longrightarrow r_1=0$, $r_2=0$.
\end{pf}

\begin{eg}
    $\mathbb{R}^2$ 上线性无关, 即两非零元夹角非零.
\end{eg}

单个非零元 $v$ 线性无关.
\begin{pf}
    $rv=0$ 且 $v\neq 0\Longrightarrow r=0$, 故 $v$ 线性无关.
\end{pf}

\begin{df}[线性相关]
    $u_1,\cdots,u_m$, 若 $\exists$ 不全为零的 $r_1,\cdots,r_m$, s.t. $r_1u_1+\cdots+r_mu_m=0$, 则称 $u_1,\cdots,u_m$ \textbf{线性相关}.
\end{df}

若 $u,v$ 线性相关, 则两者共线.
\begin{pf}
    $\exists r,t$ 不全为零, s.t. $ru+tv=0$, 无妨设 $0\neq r\in F$, 则 $ru=-tv\Longrightarrow r^{-1}ru=-r^{-1}tv\Longrightarrow u=-\frac{u}{r}v$
\end{pf}

\begin{df}[线性表示]
    $v$ 可由 $u_1,\cdots,u_n$ \textbf{线性表示} $\Longleftrightarrow\exists r_1,\cdots,r_n\in F$, s.t. $v=\sum_{i=1}^nr_iu_i$.
\end{df}

\begin{thm}[(课本定理 1.6)]\label{thm-1.6}
    $S$ 线性无关 $\Longleftrightarrow\langle S\rangle$ 中的每个向量可由 $S$ 中元素唯一地线性表示\\
    $\Longleftrightarrow S$ 中任一向量不能由 $S$ 中其余向量线性表示.
\end{thm}
\begin{pf}
    设 $S=\{u_1,\cdots,u_m\}$.

    第一个 ``$\Longrightarrow$'': $v\in\langle S\rangle$, 则 $v$ 可由 $S$ 中的元素线性表示, 即 $\exists r_1,\cdots,r_m$, s.t. $v=r_1u_1+\cdots+r_mu_m$.\\
    要证这种线性表示是唯一的, 假设 $v$ 的另一种线性表示为 $v=r_1'u_1+\cdots+r_m'u_m$.\\
    $v-v=(r_1-r_1')u_1+\cdots+(r_m-r_m')u_m=0$, 又 $\because S$ 线性无关, 即 $u_1,\cdots,u_m$ 线性无关, $\therefore r_1'=r_1$, $r_m'=r_m$, 故两种线性表示相同.

    第一个 ``$\Longleftarrow$'': $0\in\langle S\rangle$, 由于 $0u_1+\cdots+0u_m=$ 是且是 $0$ 唯一的线性表示, 故 $S$ 线性无关.

    第二个 ``$\Longrightarrow$'': 不妨假设 $u_1$ 可由 $u_2,\cdots,u_m$ 线性表示, 即 $u_1=t_2u_2+\cdots+t_mu_m$.\\
    若 $r_1u_1+\cdots+r_mu_m=0$, 则 $r_1=\cdots=r_m=0$ 或 $r_1\neq 0$, $r_2=-r_1t_2,\cdots,r_m=-r_mt_m$, 从而 $S$ 线性相关, 故假设错误, $u_1$ 不可由 $u_2,\cdots,u_m$ 线性表示.

    第二个 ``$\Longleftarrow$'': 假设 $S$ 线性相关, 则 $\exists$ 非零 $r_1,\cdots,r_m$, s.t. $r_1u_1+\cdots+r_mu_m=0$, 不妨设 $r_1$ 非零, 则 $u_1=-\frac{r_2}{r_1}u_2-\cdots-\frac{r_m}{r_1}u_m$, 即 $u_1$ 可由 $S$ 中其余向量线性表示, 矛盾, 故假设错误, $S$ 线性无关.
\end{pf}

\begin{thm}[(课本定理 1.7)]
    $\emptyset\neq S\subseteq V$, 下列等价:
    \begin{itemize}
        \item[(1)] $S$ 线性无关, 且 $V=\langle S\rangle$
        \item[(2)] $\forall v\in V$, 可用 $S$ 中元素唯一地线性表示
        \item[(3)] $S$ 是 $V$ 的极小生成集 (即 $S$ 去除任意元素都无法生成 $V$, 或 $S$ 的任意真子集都无法生成 $V$)
        \item[(4)] $S$ 是 $V$ 的极大线性无关集 (即 $S$ 增加任意元素都线性相关, $\forall u\in V$ 且 $u\notin S$, $S\cup\{u\}$ 线性相关)
    \end{itemize}
\end{thm}
\begin{pf}
    由定理 \ref{thm-1.6} 证得 (1)(2) 等价.

    设 $S=\{u_1,\cdots,u_m\}$.

    (1)$\Longrightarrow$(3): 假设 $\exists S'\subsetneq S$, s.t. $V=\langle S'\rangle$, 则 $\forall v\in S-S'\subseteq V$, $v=\sum_{i=1}^mr_iu_i$, 其中 $r_i\in F$, $u_i\in S'$, $m\in\mathbb{N}$, 即 $v$ 可由 $S$ 中的部分向量线性表示, 与 $S$ 线性无关矛盾, 故假设错误, $S$ 是 $V$ 的极小生成集.

    (3)$\Longrightarrow$(1): $S$ 为 $V$ 的生成集, 即 $V=\langle S\rangle$.\\
    假设 $S$ 线性相关, 即 $\exists r_1,\cdots,r_m$ 不全为零, s.t. $\sum_{i=1}^mr_iu_i=0$, 不妨设 $r_1\neq 0$, 则 $u_1=-\frac{r_2}{r_1}u_2+\cdots+\frac{r_m}{r_1}u_m$, 则 $S-\{u_1\}$ 仍可以生成 $V$, 矛盾, 故假设错误, $S$ 线性无关.

    (1)$\Longrightarrow$(4): 假设 $S$ 不是极大线性无关集, 则 $\exists v\in V-S$, s.t. $S\cup\{v\}$ 线性无关.\\
    又 $\because V=\langle S\rangle$, $\therefore v=\sum_{i=1}^mr_iu_i$, 其中 $r_i\in F$, $u_i\in S$, $m\in\mathbb{N}$, 即线性无关集 $S\cup\{v\}$ 中的向量 $v$ 可由其中的部分向量线性表示, 与 $S\supseteq$ 线性无关矛盾, 故假设错误, $S$ 是极大线性无关集.

    (4)$\Longrightarrow$(1): $\because S$ 是 $V$ 的极大线性无关集, $\therefore S$ 线性无关.\\
    假设 $V\neq\langle S\rangle$, $\exists v\in V-S$, s.t. $v$ 无法由 $S$ 中的元素线性表示 $\Longrightarrow S\cup\{v\}$ 为线性无关集, 与 $S$ 为最大线性无关集矛盾, 故假设错误, $V=\langle S\rangle$.

    综上, 得证.
\end{pf}

\begin{df}[基]
    任何生成向量空间 $V$ 的线性无关集. 基的阶数称为 $V$ 的\textbf{维数}, 记作 $\dim V$.
\end{df}

\begin{thm}[(课本定理 1.12)]
    向量空间的任何基都有相同的阶, 即 $\dim V$ 不依赖于基的选取.
\end{thm}

\begin{eg}
    $e_1=(1,0,\cdots,0)$, $e_2=(0,1,\cdots,0)$, $\cdots$, $e_n=(0,0,\cdots,1)$ 为 $F^n$ 的一组基.
\end{eg}
\begin{pf}
    $r_1e_1+\cdots+r_ne_n=(r_1,\cdots,r_n)=0\Longrightarrow r_1=\cdots=r_n=0$, 故 $e_1,\cdots,e_n$ 线性无关.\\
    又 $\langle\{e_1,\cdots,e_n\}\rangle=\{r_1e_1+\cdots+r_ne_n=(r_1,\cdots,r_n)\mid r_i\in F,\text{ 对 }i=1,\cdots,n\}=F$, 故得证.
\end{pf}

\textbf{找基的方法}:
\begin{itemize}
    \item[(1)] 若 $0\neq u_1\in V$, 则 $\{u_1\}$ 线性无关.
    \item[(2)] 若 $u_2\in V-\langle u_1\rangle$ 且 $u_2$ 与 $u_1$ 线性无关, 则 $\{u_1,u_2\}$ 线性无关.
    \item[(3)] 重复以上操作, 直至无法找到新的线性无关元素, 即得到极大线性无关集, 此即向量空间的基.
\end{itemize}

\begin{thm}[(课本定理 1.9)]
    线性无关集 $I\subseteq V$, $S\subseteq V$ 是 $V$ 的生成集, 且 $I\subseteq S$, 则 $\exists V$ 的基 $\mathcal{B}$, s.t. $I\subseteq\mathcal{B}\subseteq S$.
\end{thm}

\begin{df}[直和]
    \begin{itemize}
        \item[(1)] \textbf{外直和}: 若 $V_1,\cdots,V_n$ 是 $F$ 上的向量空间, $V_1\oplus\cdots\oplus V_n\equiv\{(v_1,\cdots,v_n)\mid v_i\in V_i\}$, 满足
        \begin{itemize}
            \item $(v_1,\cdots,v_n)+(u_1,\cdots,u_n)=(v_1+u_1,\cdots,v_n+u_n)$
            \item $r(v_1,\cdots,v_n)=(rv_1,\cdots,rv_n)$
        \end{itemize}
        则$V_1\oplus\cdots\oplus V_n$ 为 $F$ 的向量空间, $V_1\oplus\cdots\oplus V_n$ 为 $V_1,\cdots,V_n$ 的\textbf{外直和}.
        \item[(2)] \textbf{内直和}: $V$ 是 $F$ 上的向量空间, $V_1,\cdots,V_n$ 是 $V$ 的子空间, 若 $V=\sum_{i=1}^nV_i$, 其中 $v_i\in V_i$ 且 $V_i\cap(\cup_{j\neq i}V_j)=\{0\}$, 则称 $V$ 为 $V_1,\cdots,V_m$ 的\textbf{内直和}, 记作 $V=\bigoplus_{i=1}^nV_i$, 称 $V_i$ 为\textbf{直和项}.
    \end{itemize}
\end{df}

\textbf{内/外直和的关系}:
$V=V_1\oplus\cdots\oplus V_n$, $V_1'=\{(v_1,0,\cdots,0)\mid v_i\in V_i\}$, $\cdots$, $V_m'=\{(0,0,\cdots,v_m)\mid v_m\in V_m\}$ 是 $V$ 的子空间, 则 $V=\bigoplus_{i=1}^nV_i$ 且 $V_i'\cap(\cup_{j\neq i}V_j)=\{0\}\Longrightarrow V_i=\bigoplus_{i=1}^mV_i'$, 故内/外直和是等价的, 以下我们不明确区分内/外直和, 均用内直和.

\begin{eg}
    $\mathbb{R}^2=S_x\oplus S_y$.
\end{eg}

\begin{thm}[(课本定理 1.5)]
    $\{V_i\mid i\in J\}$ 是 $V$ 的子空间集合, $V=\sum_{i\in J}V_i$, 则下列等价:
    \begin{itemize}
        \item[(1)] $V=\bigoplus_{i\in J}V_i$
        \item[(2)] $V_i\cap(\sum_{j\neq i}V_j)\neq\{0\}$
        \item[(3)] $0=0+\cdots+0$ 是 $0$ 的唯一分解式
        \item[(4)] $V$ 中任一向量 $v$ 具有唯一分解式 $v=v_1+\cdots+v_n$, 分解式中的有限个非零元 $v_i\in V_i$ 组成的集合成为支集
    \end{itemize}
\end{thm}
\begin{pf}
    (1)$\Longleftrightarrow$(2): 由直积的定义即得证.

    (2)$\Longrightarrow$(3): 假设 $0=s_{i1}+\cdots+s_{in}$ 且 $s_{ij}$ 不全为零, 不妨设 $s_{i1}\neq 0$, 则 $V_{i1}\ni s_{i1}=-s_{i2}-\cdots-s_{ij}\in\sum_{j=2}^nV_{ij}$\\
    $\Longrightarrow s_{i_1}\in V_{i_1}\cap(\cup_{j=2}^nV_{i_j})$, $s_{i_1}\neq 0$ 与 $V_{i_1}\cap(\cup_{j=2}^nV_{i_j})=\{0\}$ 矛盾, 故假设错误, $0=0+\cdots+0$ 是 $0$ 的唯一分解式.

    (3)$\Longrightarrow$(4): $\forall v\in V$, $v=u_1+\cdots+u_n$, 其中 $u_i\in V_i$.\\
    假设 $v=w_1+\cdots+w_m$, 其中 $w_i\in V_i$.\\
    $0=v-v=u_1+\cdots+u_n-w_1-\cdots-w_n$, 将属于相同子空间的元素合并到一起, 得 $0=(u_{t_1}-w_{t_1})+\cdots+(u_{t_k}-w_{t_k})+u_{t_{k+1}}+\cdots+u_{t_n}-w_{t_{k+1}}-w_{t_m}$, 由 (2) 知 $k=n=m$ 且 $v_{t_i}=u_{t_i}$, 故 $v$ 具有唯一分解式 $v=v_1+\cdots+v_n$.

    (4)$\Longrightarrow$(2): 假设 $V_i\cap(\sum_{j\neq i}V_j)\neq\{0\}$, 则 $V_i\cap(\sum_{j\neq i}V_j)\supsetneq\{0\}$, 即 $\exists 0\neq u\in V_i\cap(\cup_{j\neq i}V_j)$,\\
    不妨设 $u\in V_1$ 且 $u\in V_2$, 则 $v=v_1+\cdots+v_n=(v_1+u)+(v_2-u)+\cdots+v_n$, 其中 $v_i\in V_i$ 且 $v_1+u\in V_1$,$v_2-u\in V_2$, $v$ 的分解式不唯一, 矛盾, 故假设错误, $V_i\cap(\sum_{j\neq i}V_j)=\{0\}$.

    综上, 得证.
\end{pf}

\begin{thm}[(课本定理 1.8)]
    $\mathcal{B}=\{v_1,\cdots,v_n\}$ 是向量空间 $V$ 的基 $\Longleftrightarrow V=\langle v_1\rangle\oplus\cdots\oplus\langle v_n\rangle$.
\end{thm}
\begin{pf}
    ``$\Longrightarrow$'': $\because\mathcal{B}$ 为 $V$ 的基, $\therefore V=\langle\mathcal{B}\rangle=\langle v_1,\cdots,v_n\rangle=\{\sum_{i=1}^nr_iv_i\mid r_i\in F\}=\langle v_1\rangle+\cdots+\langle v_n\rangle$.\\
    $\because\mathcal{B}$ 为 $V$ 的基, $\therefore v_1,\cdots,v_n$ 线性无关 $\Longrightarrow\forall 0\neq u\in\langle v_i\rangle$, $u=r_iv_i$ 且无法由 $\{v_j\mid j\neq i\}$ 线性表示 $\Longrightarrow u\notin V_i\cap(\cup_{j\neq i}V_j)$,\\
    $0=0v_i\in\langle v_i\rangle$ 且 $0=\sum_{j\neq i}0v_j\Longrightarrow 0\in V_i\cap(\cup_{j\neq i}V_j)$\\
    $\Longrightarrow V_i\cap(\cup_{j\neq i}V_j)=\{0\}$.\\
    故 $V=\langle v_1\rangle\oplus\cdots\oplus\langle v_n\rangle$.

    ``$\Longleftarrow$'': 一方面, $V=\langle v_1\rangle+\cdots+\langle v_n\rangle=\langle\mathcal{B}\rangle$;\\
    另一方面, 假设 $\{v_1,\cdots,v_n\}$ 线性相关, 则 $\exists$ 不全为零的 $r_1,\cdots,r_n$, s.t. $\sum_{i}r_iv_i=0$,\\
    不妨设 $r_i\neq 0$, 则 $r_iv_i=-\sum_{j\neq i}r_jv_j\Longrightarrow 0\neq r_iv_i\in V_i$ 且 $r_iv_i=-\sum_{j\neq i}r_jv_j\in\cup_{j\neq i}V_j\Longrightarrow r_iv_i\in V_0\cap(\cup_{j\neq i}V_j)\Longrightarrow V_0\cap(\cup_{j\neq i}V_j)\neq\{0\}$, 与直和的定义矛盾, 故假设错误, $v_1,\cdots,v_n$ 线性无关.\\
    故 $\mathcal{B}=\{v_1,\cdots,v_n\}$ 是 $V$ 的基.
\end{pf}

\begin{thm}[(课本定理 1.4)]
    $S$ 为 $V$ 的子空间, 则 $\exists V$ 的子空间 $S^c$, s.t. $V=S\oplus S^c$, 称 $S^c$ 为 $S$ 的补空间.
\end{thm}
\begin{pf}
    $\mathcal{B}_1$ 为 $S$ 的基, 则 $\mathcal{B}_1$ 为 $V$ 中的线性无关集,\\
    $\mathcal{B}_1$ 总可以扩张为 (即添加一些元素) 成 $V$ 的基, 即 $\exists\mathcal{B}_2$, s.t. $\mathcal{B}_1\cap\mathcal{B}_2=\emptyset$, $\mathcal{B}_1\cup\mathcal{B}_2$ 线性无关且 $V=\langle\mathcal{B}_1\rangle+\langle\mathcal{B}_2\rangle\Longrightarrow V=\langle\mathcal{B}_1\rangle\oplus\langle\mathcal{B}_2\rangle$, 故 $S^c=\langle\mathcal{B}\rangle$.
\end{pf}

\begin{eg}
    $\mathbb{R}^2=S_x\oplus S_y=S_l\oplus S_{l'}$, 其中 $S_l$ 和 $S_{l'}$ 分别为过原点直线 $l$ 和 $l'$ 对应的子空间, $l$ 与 $l'$ 不共线.
\end{eg}

补空间总存在, 但不唯一.

\begin{thm}[(课本定理 1.13)]
    \begin{itemize}
        \item[(1)] $\mathcal{B}$ 是 $V$ 的基, 若 $\mathcal{B}=\mathcal{B}_1\cup\mathcal{B}_2$ 且 $\mathcal{B}_1\cap\mathcal{B}_2=\emptyset$, 则 $V=\langle\mathcal{B}_1\rangle\oplus\langle\mathcal{B}_2\rangle$.
        \item[(2)] $V=S\oplus T$, 若 $\mathcal{B}_1$ 是 $S$ 的基, $\mathcal{B}_2$ 是 $T$ 的基, 则 $\mathcal{B}_1\cap\mathcal{B}_2=\emptyset$, $\mathcal{B}_1\cup\mathcal{B}_2$ 是 $V$ 的基.
    \end{itemize}
\end{thm}
\begin{pf}
    \begin{itemize}
        \item[(1)] $\because\mathcal{B}$ 是 $V$ 的基, $\therefore\forall u\in V$, $u=\sum_{i=1}^kr_iv_i$, 其中 $r_i\in F$, $v_i\in\mathcal{B}$, $k\in\mathbb{N}$.\\
        $\langle\mathcal{B}_1\rangle=\{\sum_{i=1}^nr_iv_i\mid r_i\in F,v_i\in\mathcal{B}_1,n\in\mathbb{N}\}$, $\langle\mathcal{B}_2\rangle=\{\sum_{i=1}^nr_iv_i\mid r_i\in F,v_i\in\mathcal{B}_2,n\in\mathbb{N}\}$.\\
        $u=\sum_{i=1}^tr_iv_i+\sum_{i=t+1}^kr_iv_i$, 其中 $v_1,\cdots,v_t\in\mathcal{B}_1$, $v_{t+1},\cdots,v_k\in\mathcal{B}_2\Longrightarrow V=\langle\mathcal{B}_1\rangle+\langle\mathcal{B}_2\rangle$.

        $\forall u\in\langle\mathcal{B}_1\rangle\cap\langle\mathcal{B}_2\rangle$,$u\in\langle\mathcal{B}_1\rangle\Longrightarrow u=\sum_{i=1}^nr_iv_i$, 其中 $r_i\in F$, $v_i\in\mathcal{B}_1$,\\
        且 $u\in\langle\mathcal{B}_2\rangle\Longrightarrow u=\sum_{i=1}^nl_iw_i$, 其中 $l_i\in F$, $w_i\in\mathcal{B}_2$\\
        $\Longrightarrow 0=u-u=\sum r_iv_i-\sum l_iw_i$,\\
        又 $\because\mathcal{B}$ 为基, $\mathcal{B}=\mathcal{B}_1\cup\mathcal{B}_2$ 且 $\mathcal{B}_1\cap\mathcal{B}_2=\emptyset$, $\therefore r_i,w_i$ 线性无关 $\Longrightarrow r_i=l_i=0$, $\forall i$\\
        $\Longrightarrow u=0$.

        综上, $V=\langle\mathcal{B}_1\rangle\oplus\langle\mathcal{B}_2\rangle$.
        \item[(2)] $V=S\oplus T\Longleftrightarrow V=S+T$ 且 $S\cap T=\{0\}$.\\
        假设 $v\in\mathcal{B}_1\cap\mathcal{B}_2$, 则 $v\neq 0$, $\langle v\rangle=S\cap T$, 与 $S\cap T=\{0\}$ 矛盾, 故假设错误, $\mathcal{B}_1\cap\mathcal{B}_2=\emptyset$.

        $\because V=S+T$, $\therefore\forall u\in V$, $u=u_1+u_2$, 其中 $u_1\in S$, $u_2\in T$,\\
        $\because\mathcal{B}_1$ 是 $S$ 的基, $\mathcal{B}_2$ 是 $T$ 的基, $\therefore u_1=\sum_{i=1}^kr_iv_i$, $u_2=\sum_{i=k+1}^n$, 其中 $r_i\in F$, 对 $i=1,\cdots,k$, $v_i\in\mathcal{B}_1$, 对 $i=k+1,\cdots,n$, $v_i\in\mathcal{B}_2$\\
        $\Longrightarrow u=\sum_{i=1}^mr_iv_i$, 其中 $r_i\in F$, $v_i\in\mathcal{B}_1\cap\mathcal{B}_2$, 即 $V=\langle\mathcal{B}_1\cup\mathcal{B}_2\rangle$.\\
        假设 $\mathcal{B}_1\cup\mathcal{B}_2$ 线性相关, 则 $\exists r_i\in F$ 不全为零, $\sum_{i=1}^nr_iv_i=\sum_{i=1}^kr_iv_i+\sum_{i=k+1}^nr_iv_i=0$, 其中 $r_i\in F$, 对 $i=1,\cdots,k$, $v_i\in\mathcal{B}_1$, 对 $i=k+1,\cdots,n$, $v_i\in\mathcal{B}_2$,\\
        $\because\mathcal{B}_1$ 和 $\mathcal{B}_2$ 为基, $\therefore\mathcal{B}_1$ 和 $\mathcal{B}_2$ 线性无关 $\Longrightarrow\sum_{i=1}^kr_iv_i\neq 0$, $\sum_{i=k+1}^nr_iv_i\neq 0$, 与 $0=0+\cdots+0$ 是 $0$ 的唯一分解式矛盾, 故假设错误, $\mathcal{B}_1\cup\mathcal{B}_2$ 线性无关 $\Longrightarrow\mathcal{B}_1\cup\mathcal{B}_2$ 是 $V$ 的基.
    \end{itemize}
\end{pf}

\begin{thm}[(课本定理 1.14)]
    $S,T$ 是 $V$ 的子空间, $\dim S+\dim T=\dim(S\cap T)+\dim(S+T)$. 特别地, 若 $T$ 是 $S$ 的补空间, 则 $\dim S+\dim T=\dim(S\oplus T)$.
\end{thm}
\begin{pf}
    设 $S\cap T$ 的基为 $\mathcal{A}$,\\
    $\because S\cap T$ 为 $S$ 的子空间, $\therefore$ 可将 $\mathcal{A}$ 扩张成 $S$ 的基 $\mathcal{A}\cup\mathcal{B}$,\\
    $\because S\cap T$ 为 $T$ 的子空间, $\therefore$ 可将 $\mathcal{A}$ 扩张成 $T$ 的基 $\mathcal{A}\cup\mathcal{C}$.

    接下来需要用到这样一个事实: $\mathcal{A}\cup\mathcal{B}\cup\mathcal{C}$ 是 $S+T$ 的基. 所以先来证明它:
    \begin{pf}
        $\forall w\in S+T$, $w=u+v$, 其中 $u\in S$, $v\in T\Longrightarrow u\in \langle\mathcal{A}\cup\mathcal{B}\rangle$, $v\in\langle\mathcal{A}+\mathcal{C}\rangle$, 故 $\langle\mathcal{A}\cup\mathcal{B}\cup\mathcal{C}\rangle=S+T$.

        不妨设 $\sum_{i=1}^nr_iv_i=0$, 其中 $v_i\in\mathcal{A}\cup\mathcal{B}\cup\mathcal{C}$.\\
        设 $v_1,\cdots,v_k\in\mathcal{A}$, 则 $\sum_{i=1}^kr_iv_i+\sum_{i=k+1}^nr_iv_i=0$,\\
        令 $x=\sum_{i=1}^kr_iv_i$, 则 $x=\sum_{i=1}^kr_iv_i\in\langle\mathcal{A}\rangle$ 且 $x=-\sum_{i=k+1}^nr_iv_i\in\langle\mathcal{B}\cup\mathcal{C}\rangle\Longrightarrow x\in\langle\mathcal{A}\rangle\cap\langle\mathcal{B}\cup\mathcal{C}\rangle=(S-T)\cap T=\emptyset$.\\
        $\because x\in\langle\mathcal{B}\rangle$, $\therefore x\in\mathcal{S}$, 又 $\because x\in \langle\mathcal{B}\cup\mathcal{C}\rangle$, $\therefore x\in T\Longrightarrow x\in S\cap T=\langle\mathcal{B}\rangle$.
        $\Longrightarrow x\in\langle\mathcal{A}\rangle\cap\langle\mathcal{B}\rangle\Longrightarrow x=0$.\\
        又 $\because\mathcal{A}$ 和 $\mathcal{B}\cup\mathcal{C}$ 线性独立, 故 $\forall i$, $r_i=0\Longrightarrow\mathcal{A}\cup\mathcal{B}\cup\mathcal{C}$ 线性无关.

        综上, $\mathcal{A}\cup\mathcal{B}\cup\mathcal{C}$ 是 $S+T$ 的基.
    \end{pf}

    故 $$\dim S+\dim T=\abs{\mathcal{A}\cup\mathcal{B}}+\abs{\mathcal{B}\cup\mathcal{C}}=\abs{\mathcal{A}}+\abs{\mathcal{B}}+\abs{\mathcal{B}}+\abs{\mathcal{C}}=\abs{\mathcal{A}}+\abs{\mathcal{B}}+\abs{\mathcal{C}}+\dim(S\cap T)=\dim(S+T)+\dim(S\cap T).$$
\end{pf}
\ifx\allfiles\undefined
\end{document}
\fi