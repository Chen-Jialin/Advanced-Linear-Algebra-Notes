% !Tex program = pdflatex
% 第 4 章: 模 I: 基本性质 Chap 4: Modules I: Basic Properties
\ifx\allfiles\undefined
\documentclass{note}
\setcounter{chapter}{+3}
\begin{document}
\fi
\chapter{模 I: 基本性质}
\section{模}
\begin{df}[模]
    $R$ 为有单位元交换环, $(M,+)$ 为交换群, 数乘 $:R\times M\rightarrow M$, $(r,m)\mapsto m$, 若满足
    \begin{itemize}
        \item[(1)] $(r+t)m=rm+tm$,
        \item[(2)] $(rt)m=r(tm)$,
        \item[(3)] $r(m_1+m_2)=rm_1+rm_2$,
        \item[(4)] $1m=m$,
    \end{itemize}
    则称 $M$ 为 $R$ 上的\textbf{模}, 记作 $R-\module\equiv\{R$ 上的模$\}$.
\end{df}

$\because$ 域是一种特殊的环, $\therefore$ 向量空间是一种特殊的模.

$0m=0$.
\begin{pf}
    $0m+0m=(0+0)m=0m\Longrightarrow 0m=0$.
\end{pf}

$r0=0$.
\begin{pf}
    $r0+r0=r(0+0)=r0\Longrightarrow r0=0$.
\end{pf}

$(-r)m=r(-m)=-(rm)$.
\begin{pf}
    $(-r)m+rm=(-r+r)m=0m=0\Longrightarrow(-r)m=-rm$.\\
    $r(-m)+rm=r(m+(-m))=r0=0\Longrightarrow r(-m)=-rm$.
\end{pf}

$\forall r\in R$, 可构造映射 $\bar{r}:M\rightarrow M$, $m\mapsto rm$.\\
$\bar{r}$ 为 $M$ 上的群同态, 又称\textbf{自同态}, 记作 $\bar{r}\in\End(M)\equiv\{M$ 上的自同态$\}$.\\
$\End(M)$ 关于同态的加法、复合成环, 其单位元为 $M$ 上的恒等映射, 记作 $1_M$, 故还可构造映射 $\phi:R\rightarrow\End(M)$, $r\mapsto\bar{r}$.
\begin{pf}
    $\bar{r}(m+n)=r(m+n)=rm+rn=\bar{r}(m)+\bar{r}(n)$, 即映射 $\bar{r}$ 下保持运算结构, 故得证.
\end{pf}

\begin{eg}
    在交换群 $(G,+)$ 上定义数乘 $\alpha:\mathbb{Z}\times G\rightarrow G$, $(n,a)\mapsto na$, 其中 $1a=a$, $2a=a+a$, $\cdots$, $na=\overbrace{a+\cdots+a}^{\text{$n$ 个 $a$ 相加}}$, $-a=-1a$, $-2a=(-a)+(-a)$, $-na=\overbrace{(-a)+\cdots+(-a)}^{\text{$n$ 个 $(-a)$ 相加}}$. $na$ 的定义依赖于 $G$ 中的运算, 而运算的本质是卡氏积至原集合的映射, 有唯一的结果, 故该数乘为映射. 此时交换群及其数乘满足
    \begin{itemize}
        \item[(1)] $(n+m)a=\overbrace{a+\cdots+a}^{\text{$(n+m)$ 个 $a$ 相加}}=\overbrace{a+\cdots+a}^{\text{$n$ 个 $a$ 相加}}+\overbrace{a+\cdots+a}^{\text{$m$ 个 $a$ 相加}}=na+ma$,
        \item[(2)] $(nm)a=\overbrace{a+\cdots+a}^{\text{$nm$ 个 $a$ 相加}}=\overbrace{\overbrace{a+\cdots+a}^{\text{$m$ 个 $a$ 相加}}+\cdots+\overbrace{a+\cdots+a}^{\text{$m$ 个 $a$ 相加}}}^{\text{$n$ 组}}=\overbrace{ma+\cdots+ma}^{\text{$n$ 个 $ma$ 相加}}=n(ma)$,
        \item[(3)] $n(a+b)=\overbrace{(a+b)+\cdots+(a+b)}^{\text{$n$ 个 $(a+b)$ 相加}}=\overbrace{a+\cdots+a}^{\text{$n$ 个 $a$ 相加}}+\overbrace{b+\cdots+b}^{\text{$n$ 个 $b$ 相加}}=na+nb$,
        \item[(4)] 由定义显然有 $1a=a$,
    \end{itemize}
    故 $\forall$ 交换群 $(G,+)$, $G\in\mathbb{Z}-\module$
\end{eg}

\begin{eg}
    $R\in R-\module$, 其中的数乘即 $R$ 中的乘法.
\end{eg}

\begin{eg}\label{module Zp}
    $\mathbb{Z}_p=\frac{\mathbb{Z}}{p}=\{[0],\cdots,[p-1]\}$, $(\mathbb{Z}_p,+)$ 是交换群, 故 $\mathbb{Z}_p\in\mathbb{Z}-\module$.

    $\mathbb{Z}_6=\{[0],[1],[2],[3],[4],[5]\}$, $n[k]=\overbrace{[k]+\cdots+[k]}^{\text{$n$ 个 $[k]$ 相加}}=[nk]$.\\
    注意到 $[2]\neq[0]$, $3\neq 0$, 但 $3[2]=[6]=[0]$, 即非零元素的卡氏积在数乘映射下得到零元素, 这意味着非零的单个元素不再线性无关.

    实际上, $\mathbb{Z}_p$ 中无线性无关元素.
\end{eg}

\begin{eg}
    $R^n=\{(r_1,\cdots,r_n)\mid r_i\in R\}\in R-\module$, 其中 $(r_1,\cdots,r_n)+(l_1,\cdots,l_n)=(r_1+l_1,\cdots,r_n+l_n)$, $r(r_1,\cdots,r_n)=(rr_1,\cdots,rr_n)$.
\end{eg}

\section{子模}
\begin{df}[子模]
    $\emptyset\neq S\subseteq M$, 若在 $M$ 的运算下, $S$ 是 $R$ 上的模, 则称 $S$ 为 $M$ 的\textbf{子模}.
\end{df}

\begin{thm}[子模的判定方法 (课本定理 4.1)]
    $\emptyset\neq S\subseteq M$ 是 $M$ 的子模 $\Longleftrightarrow\forall u,v\in S$, $\forall r,t\in R$, $ru+tv\in S$ (即线性运算封闭).
\end{thm}

\begin{thm}[(课本定理 4.2)]
    $S,T\subseteq M$ 是 $M$ 的子模, 则 $S\cap T$ 为 $M$ 的子模, $S+T\equiv\{u+v\mid u\in S,v\in T\}$ 为 $M$ 的子模.
\end{thm}

\begin{thm}
    $R\in R-\module $, $R$ 的子模即 $R$ 上的理想.
\end{thm}
\begin{pf}
    设 $S$ 为 $R$ 的子模, 则
    \begin{itemize}
        \item[(1)] $\emptyset\neq S\subseteq R$,
        \item[(2)] $\forall u,v\in S$, $\forall r,t\in R$, $ru+tv\in S$.\\
        特别地, 令 $r=1$, $t=-1$, 得 $u-v\in S$, 令 $t=0$, 得 $ru\in S$,
    \end{itemize}
    故 $S$ 为 $R$ 的理想.
\end{pf}

\section{生成模}
\begin{df}[生成子模和生成集]
    $\emptyset\neq S\subseteq M\in R-\module$, $S$ 的\textbf{生成子模}为 $\langle\langle S\rangle\rangle\equiv$ 包含 $S$ 的最小子模 $\equiv$ 包含 $S$ 的所有子模的交 $=\{\sum_{i=1}^nr_iu_i\mid r_i\in R,u_i\in S,n\in\mathbb{Z}^+\}$, 其中称 $S$ 为\textbf{生成集}.
\end{df}

$\because M=\langle\langle M\rangle\rangle$, $\therefore\forall M\in R-\module$, 都有生成集.

\begin{df}[有限生成模]
    由有限个元素生成的模.
\end{df}

\begin{df}[循环模]
    由一个元素生成的模.
\end{df}

\begin{eg}
    $\because R=\langle\langle 1\rangle\rangle=\{r1\mid r\in R\}$, $\therefore R\in R-\module$ 是一个循环模.
\end{eg}

有限生成模的子模未必是有限生成的, 即有限生成的性质未必会由模遗传至其子模.
\begin{eg}
    多项式环 $R=F[x_1,\cdots,x_n,\cdots]\equiv\left\{\sum_{k_1=0}^{N_1}\cdots\sum_{k_n}^{N_n}a_{i_1,\cdots,i_n}x_{i_1}^{k_1}\cdots x_{i_n}^{k_n}\mid a_{i_1\cdots i_n}\in F,N_i\in\mathbb{Z}^+\right\}$, $R\in R-\module$ 且 $R=\langle\langle 1\rangle\rangle$.\\
    令子模 $S$ 为 $R$ 的常数项为零的多项式构成的子集, 则 $S$ 为 $R$ 的子模且 $S=\langle\langle x_1,x_2,\cdots\rangle\rangle$, 即 $S$ 并非是有限生成的.
\end{eg}

\begin{df}[线性无关]
    $\emptyset\neq S\subseteq M$, 若 $\sum_{i=1}^nr_iu_i=0$ 其中 $u_i\in S$, $r_i\in R\forall i\Longrightarrow r_1=\cdots=r_n=0$, 则称 $S$ 线性无关.
\end{df}

在模中, 线性无关元素未必存在, 如例 \ref{module Zp} 中 $\mathbb{Z}_p$ 无线性无关元素.

在向量空间中, 我们有: $u,v$ 线性相关 $\Longleftrightarrow\exists$ 不全为零的 $r,t\in R$, s.t. $ru+tv=0$, 无妨设 $r\neq 0$, 则 $ru=-tv\Longrightarrow u=-\frac{t}{r}v$.

在模中, 上述说法未必成立: $u,v$ 线性相关 $\Longleftrightarrow\exists$ 不全为零的 $r,t$, s.t. $ru+tv=0$, (无妨设 $r\neq 0$,) 则 $ru=-tv$, 但由于未必能找到 $r$ 的逆元, 所以未必有 $u=-\frac{t}{r}v$. 故在模中, 线性相关元素未必能相互表示, 即一个线性相关元素未必能由与其线性相关的元素线性表示.

\section{自由模}
\begin{df}[自由模]
    $M\in R-\module$, $M=\langle\langle\mathcal{B}\rangle\rangle$ 且 $\mathcal{B}$ 线性无关, 则称 $M$ 为\textbf{自由模}, $\mathcal{B}$ 为 $M$ 的\textbf{基}.
\end{df}

\begin{thm}[(课本定理 4.3)]
    $\emptyset\neq\mathcal{B}\subseteq M$ 是 $M$ 的基, 则 $\forall v\in M$, $v$ 可由 $\mathcal{B}$ 中的元素唯一地线性表示.
\end{thm}

\begin{thm}[(课本定理 4.4)]
     $\mathcal{B}$ 是 $M$ 的基 $\Longleftrightarrow\mathcal{B}$ 为 $M$ 的极小生成集\uline{且}为 $M$ 的极大线性无关集.
\end{thm}

\begin{eg}
    $\mathbb{Z}_6=\{[0],[1],[2],[3],[4],[5]\}$.\\
    $\because 0[1]=[0]$, $1[1]=[1]$, $2[1]=[2]$, $3[1]=[3]$, $4[1]=[4]$, $5[1]=[5]$, $\therefore\mathbb{Z}_6=\langle\langle[1]\rangle\rangle$.\\
    $0[5]=[0]$, $1[5]=[5]$, $2[5]=[10]=[4]$, $3[5]=[15]=[3]$, $4[5]=[20]=[2]$, $5[5]=[25]=[1]$, $\therefore\mathbb{Z}_6=\langle\langle[6]\rangle\rangle$.\\
    故 $\mathbb{Z}_6$ 的表示不唯一.
\end{eg}

$M\in R-\module$, 但 $M$ 的子模未必自由.
\begin{eg}
    $R=\mathbb{Z}\times\mathbb{Z}=\{(n,m)\mid n,m\in\mathbb{Z}\}$, 其中 $(n,m)(k,l)=(nk,ml)$, $(n,m)+(k,l)=(n+k,m+l)$ 仅为交换环 (而非域), $R\in R-\module$, $R=\langle\langle(1,1)\rangle\rangle=\{r(1,1)\mid r\in R=\mathbb{Z}\times\mathbb{Z}\}$, $\therefore R$ 自由.\\
    但子模 $S=\mathbb{Z}\times\{0\}=\{(n,0)\mid n\in\mathbb{Z}\}$, $\because\forall n\neq 0$, $(n,0)(0,1)=(0,0)$, $\therefore$ 无线性无关元, 从而非自由.
\end{eg}

\section{模同态}
\begin{df}[模同态]
    $M,N\in R-\module$, 映射 $\tau:M\rightarrow N$, 若 $\forall u,v\in M$, $\forall r,t\in R$, $\tau(ru+tv)=r\tau(u)+t\tau(v)$, 则 $\tau$ 为 $M$ 到 $N$ 的\textbf{模同态}, 记作 $\tau\in\hom(M,N)=\{M$ 到 $N$ 的模同态$\}$.
\end{df}

取 $r=t=1$, 则 $\forall u,v\in M$, $\tau(u+v)=\tau(u)+\tau(v)$, 故 $\tau$ 为群同态.

\begin{thm}[(课本定理 4.6)]
    \begin{itemize}
        \item[(1)] $\ker\tau\equiv\{v\in M\mid\tau(v)=0\}$ 是 $M$ 的子模. $\tau$ 单射 $\Longleftrightarrow\ker\tau=\{0\}$.
        \item[(2)] $\im\tau\equiv\{\tau(v)\mid v\in M\}$ 是 $N$ 的子模. $\tau$ 满射 $\Longleftrightarrow\im\tau=N$.
    \end{itemize}
\end{thm}

\section{商环}
\begin{df}[商模]
    $S$ 是 $M$ 的子模, \textbf{商模} $\frac{M}{S}\equiv\{[v]\mid v\in M\}$.
\end{df}

$\because$ 结果与代表元选取无关, $\therefore[u]+[v]=[u+v]$, $r[u]=[ru]$ 是合法运算.

\begin{center}
    \begin{tikzpicture}
        \node(1)at(0,2){$M$};
        \node(2)at(2,2){$N$};
        \node(3)at(0,0){$\frac{M}{S}$};
        \node(4)at(-1,3){$v$};
        \node(5)at(4,3){$\tau[v]$};
        \node(6)at(-1,-2){$[v]$};
        \draw[->](1)--(2)node[midway]{$\tau$};
        \draw[->](1)--(3)node[midway]{$\Pi_S$};
        \draw[dashed,->](3)--(2)node[midway]{$\tau'$};
        \draw[|->](4)--(5);
        \draw[|->](4)--(6);
        \draw[dashed,|->](6)--(5);
    \end{tikzpicture}
\end{center}
$\Pi_S:M\rightarrow\frac{M}{S}$, $v\mapsto[v]$, 且满足
\begin{itemize}
    \item[(1)] $\Pi_S$ 满射.
    \item[(2)] $\ker\Pi_S=S$.
\end{itemize}

\section{同构定理}
\begin{thm}[第一同态基本定理]
    若 $S\subseteq\ker\tau$, 则 $\exists!\tau':\frac{M}{S}\rightarrow N$, s.t. $\tau=\tau'\circ\Pi_S$, 即 $\tau(v)=\tau'([v])$, 且 $\ker\tau'=\frac{\ker\tau}{S}$, $\im\tau'=\im\tau$.
\end{thm}

\begin{thm}[第一同构基本定理]
    若 $S=\ker\tau$, 则 $\ker\tau'=\frac{\ker\tau}{S}=\{[0]\}$, 即 $\tau'$ 单射.\\
    $\frac{M}{\ker\tau}=\frac{M}{S}\approx\im\tau$.
\end{thm}
\ifx\allfiles\undefined
\end{document}
\fi
