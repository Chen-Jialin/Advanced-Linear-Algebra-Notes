% !Tex program = pdflatex
% 第 12 章: 度量空间 Chap 12: Metric Spaces
\ifx\allfiles\undefined
\documentclass{note}
\begin{document}
\fi
\setcounter{chapter}{+11}
\chapter{度量空间}
\begin{df}[度量和度量空间]
    对集合 $M$, 映射 $d(,):M\times M\rightarrow\mathbb{R}$, 若满足
    \begin{itemize}
        \item[(1)] \textbf{正定}: $d(u,v)\geq 0$, 且 $d(u,v)=0\Longleftrightarrow u=v$,
        \item[(2)] \textbf{对称}: $d(u,v)=d(v,u)$,
        \item[(3)] \textbf{三角不等式}: $d(u,v)\leq d(u,v)+d(u,v)$,
    \end{itemize}
    则称 $d$ 为 $M$ 上的一个\textbf{度量}, 称 $(M,d)$ 为\textbf{度量空间}.
\end{df}

对任一集合均可定义度量, 如 $d(u,v)=\left\{\begin{array}{ll}
    1,&u\neq v,\\
    0,&u=v,
\end{array}\right.$ 且度量不唯一.

\begin{df}[子度量空间]
    度量空间的非空子集.
\end{df}

\section{开集和闭集}
对度量空间 $(M,d)$, $x_0\in M$, $r>0$, 可定义:
\begin{df}[开球]
    $B(x_0,r)\equiv\{y\in M\mid d(y,x_0)<r\}$.
\end{df}

\begin{df}[闭球]
    $\bar{B}(x_0,r)\equiv\{y\in M\mid d(y,x_0)\leq r\}$.
\end{df}

\begin{df}[球面]
    $S(x_0,r)\equiv\{y\in M\mid d(y,x_0)=r\}$.
\end{df}

\begin{df}[开集]
    $S\subseteq M$, $\forall x_0\in S$, $\exists r>0$, s.t. $B(x_0,r)\subseteq S$, 则称 $S$ 为\textbf{开集}.
\end{df}

\begin{df}[闭集]
    $T\subseteq M$, $T^c=M\backslash T$ 是开集, 则 $T$ 为\textbf{闭集}.
\end{df}

\begin{df}[开邻域]
    包含 $x_0$ 的任何开集称 $x_0$ 的\textbf{开邻域}.
\end{df}

开球为开集, 闭球为闭集.
\begin{pf}
    $\forall$ 开球 $B(x_0,r_0)=\{y\in M\mid d(y,x_0)<r_0\}$, $\forall x\in B(x_0,r_0)$, $\exists r=\frac{1}{2}(r_0-d(x,x_0))>0$, s.t. $B(x,r)\subseteq B(x_0,r_0)\Longrightarrow$ 开球 $B(x_0,r_0)$.

    $\forall$ 闭球 $\bar{B}(x_0,r_0)=\{y\in M\mid d(y,x_0)\leq r_0\}$, $\forall x\in(\bar{B}(x_0,r_0))^c=\{y\in M\mid d(y,x_0)>r_0\}$, $\exists r=\frac{1}{2}(d(x,x_0)-r)>0$, s.t. $B(x,r)\subseteq(\bar{B}(x_0,r_0))^c\Longrightarrow(\bar{B}(x_0,r_0))^c$ 为开集, 即 $\bar{B}(x_0,r_0)$ 为闭集.
\end{pf}

\begin{eg}
    设 $\mathbb{R}$ 上的度量 $d(r,t)=\abs{r-t}$. 对点 $x_0$,\\
    $B(x_0,r)=\{y\in\mathbb{R}\mid\abs{y-x_0}<r\}=(x_0-r,x_0+r)$,\\
    $\bar{B}(x_0,r)=[x_0-r,x_0+r]$,\\
    $S(x_0,r)=\{x_0-r,x_0+r\}$.\\
    开区间 $(a,b)$ 为开集.
    \begin{pf}
        $\forall x\in(a,b)$, 取 $r=\frac{1}{2}\min\{\abs{x-a},\abs{x-b}\}>0$, $B(x,r)=(x-r,x+r)\subseteq(a,b)$, 故得证.
    \end{pf}
    闭区间 $[a,b]$ 为闭集.\\
    $(a,b]$ 既非开集也非闭集.
\end{eg}

\begin{thm}[(课本定理 12.1)]
    $\mathcal{O}=\{M\text{ 上的开集}\}$, 则
    \begin{itemize}
        \item[(1)] $\emptyset,M\in\mathcal{O}$.
        \item[(2)] 有限个开集的交仍为开集: $S,T\in\mathcal{O}$, 则 $S\cap T\in\mathcal{O}$.
        \item[(3)] $S_i\in\mathcal{O}$, 则 $\cup_{i\in K}S_i\in\mathcal{O}$ ($S_i$ 可以是无限个).
    \end{itemize}
\end{thm}
\begin{pf}
    \begin{itemize}
        \item[(1)] 显然.
        \item[(2)] $x_0\in S\cap T\Longleftrightarrow x_0\in S$, $x_0\in T$.\\
        $\because S$ 为开集, $\therefore\exists r_1>0$, s.t. $B(x_0,r_1)\subseteq S$;\\
        $\because T$ 为开集, $\therefore\exists r_2>0$, s.t. $B(x_0,r_2)\subseteq T$.\\
        令 $r=\min\{r_1,r_2\}$, 则 $B(x_0,r)\subseteq S$ 且 $B(x_0,r)\subseteq T\Longrightarrow B(x_0,r)\subseteq S\cap T\Longrightarrow S\cap T$ 开, 即 $S\cap T\in\mathcal{O}$.
        \item[(3)] $x_0\in\cup_iS_i\Longleftrightarrow\exists i$, s.t. $x_0\in S_i$.\\
        $\because S_i$ 为开集, $\therefore\exists r>0$, s.t. $B(x_0,r)\subseteq S_i\subseteq\cup_iS_i\Longrightarrow\cup_iS_i$ 开, 即 $\cup_iS_i\in\mathcal{O}$.
    \end{itemize}
\end{pf}

\begin{eg}[无穷多个开集的交未必开]
    $S_n=\left(-\frac{1}{n},\frac{1}{n}\right)$.\\
    $\cap_iS_i=\lim_{n\rightarrow\infty}S_n=\{0\}$ 闭.
\end{eg}

单点集为闭集.
\begin{pf}
    单点集 $S=\{a\}$, 补集 $S^c=M\backslash\{a\}$.\\
    $\forall x\in S^c$, 则 $x\neq a$, $d(x,a)>0$.\\
    取 $r=\frac{1}{2}d(x,a)$, 则 $a\neq B(x,r)\Longrightarrow B(x,r)\subseteq S^c\Longrightarrow S^c$ 开, 故 $S$ 闭.
\end{pf}

有限点集为闭集.
\begin{pf}
    有限点集 $S=\{a_1,\cdots,a_n\}$, 补集 $S^{c}=M\backslash\{a_1,\cdots,a_n\}$.\\
    $\forall x\in S^c$, 则 $x\neq a_1,\cdots,a_n\Longrightarrow\forall i$, $d(x,a_i)>0$.\\
    取 $r=\frac{1}{2}\min_i\{d(x,a_i)\}$, 则 $\forall a_i$, $a_i\neq B(x,r)\Longrightarrow S^c$ 开, 故 $S$ 闭.
\end{pf}

\begin{df}[拓扑和拓扑空间]
    集合 $X\neq\emptyset$, $\mathcal{O}$ 是 $X$ 的一些子集构成的簇\footnote{可理解为``集合的集合'', 此处为避免逻辑循环, 故名之}, 若
    \begin{itemize}
        \item[(1)] $\emptyset,X\in\mathcal{O}$,
        \item[(2)] $S,T\in\mathcal{O}$, 则 $S\cap T\in\mathcal{O}$,
        \item[(3)] $\{S_i\in\mathcal{O}\mid i\in K\}$, 则 $\cup_{i\in K}S_i\in\mathcal{O}$,
    \end{itemize}
    则称 $\mathcal{O}$ 为 $X$ 上的一个\textbf{拓扑}, 称 $(X,\mathcal{O})$ 为\textbf{拓扑空间}, 称 $\mathcal{O}$ 中的集合为 $X$ 上的\textbf{开集}.
\end{df}

故确定开集 $\Longleftrightarrow$ 确定拓扑.

\section{度量空间的收敛性}
\begin{df}[收敛和极限]
    集合 $M$ 中序列 $(x_n)$, $x\in M$, 若 $\lim_{n\rightarrow\infty}d(x_n,x)=0$, 则称序列 $(x_n)$ \textbf{收敛}于 $x$, 记作 $x_n\rightarrow x$, 称 $x$ 为序列 $(x_n)$ 的\textbf{极限}.
\end{df}

\begin{eg}
    $(r_n)$ 为 $\mathbb{R}$ 上的序列, $\lim_{n\rightarrow\infty}r_n=0$, 即 $r_n\rightarrow r$, 即 $\forall\epsilon>0$, $\exists N>0$, 当 $n>N$ 时, $d(r_n,r)<\epsilon\Longleftrightarrow r_n\in B(r,\epsilon)$.
\end{eg}

\textbf{收敛的性质}:
\begin{itemize}
    \item[(1)] $\forall r>0$, $B(x,r)$ 中包含 $(x_n)$ 中无穷个元素.
    \item[(2)] 有限点列非序列, 无需考虑极限.
    \item[(3)] 常序列收敛, $(x_n=x_0)\rightarrow x_0$.
    \item[(4)] 对给定序列, 若 $\exists$ 极限, 则极限唯一.
\end{itemize}

\begin{thm}[(课本定理 12.2)]
    闭集关于收敛封闭. $S$ 闭 $\Longleftrightarrow S$ 中序列 $(x_n)\rightarrow x\in M$, 则 $x\in S$.
\end{thm}
\begin{pf}
    ``$\Longrightarrow$'': 取 $S$ 中序列 $(x_n)$ 且 $(x_n)\rightarrow x\in M$.\\
    假设 $x\notin S$, 则 $x\in S^c=M\backslash S$.\\
    $\because S$ 闭, $\therefore S^c$ 开 $\Longrightarrow\exists r>0$, s.t. $B(x,r)\subseteq S^c$, 即 $B(x,r)\cap S=\emptyset$, 故 $S$ 中序列 $(x_n)\notin B(x,r)$.\\
    又 $\because x_n\rightarrow x$, $\therefore\exists N_r>0$, s.t. 当 $n>N_r$ 时, $x_n\in B(x,r)$, 矛盾, 故假设错误, $x\in S$.

    ``$\Longleftarrow$'': 假设 $S$ 非闭, 则 $S^c$ 非开
    $\Longrightarrow\exists x_0\in S^c$, s.t. $\forall r>0$, $B(x_0,r)\nsubseteq S^c$.\\
    特别地, 令 $r=1$, 则 $\exists x_1\in B(x_0,r)$, s.t. $x_1\notin S^c$, 即 $x_1\in S$,\\
    $\cdots$, 令 $r=\frac{1}{n}$, 则 $\exists x_n\in B(x_0,r)$, s.t. $x_n\notin S^c$, 即 $x_n\in S$,\\
    $\cdots\Longrightarrow x_n\rightarrow x_0$.\\
    又 $\because(x_n)\in S$, $\therefore$ 由题设, $x_0\in S$, 矛盾, 故假设错误, $S$ 闭.

    综上, 得证.
\end{pf}

\section{集合的闭包}
\begin{df}[闭包]
    $S\subseteq M$, 称包含 $S$ 的最小闭集或包含 $S$ 的所有闭集的交为 $S$ 的\textbf{闭包}, 记作 $\cl(S)$.
\end{df}

给定 $S$, 必 $\exists$ 其闭包.

\begin{df}[极限点(/聚点)]
    $\emptyset\neq S\subseteq M$, $x\in M$, 若 $\forall r>0$, $B(x,r)\cap S$ 包含异于 $x$ 的点, 则称 $x$ 为 $S$ 的\textbf{极限点}或\textbf{聚点}, $S$ 对应的极限点的集合记作 $\lp(S)$.
\end{df}

\begin{eg}
    $(a,b)\subseteq\mathbb{R}$, 则 $\lp((a,b))=[a,b]$.\\
    $c\notin[a,b]$, $S=(a,b)\cup\{c\}$, 则 $\lp(S)=[a,b]$.
\end{eg}

\begin{thm}[(课本定理 12.3)]
    \begin{itemize}
        \item[(1)] $x\in\lp(S)\Longleftrightarrow\exists$ 序列 $(x_n)\in S$, s.t. $x_n\neq x\forall n$ 且 $x_n\rightarrow x$.
        \item[(2)] $S$ 闭 $\Longleftrightarrow\lp(S)\subseteq S$.
        \item[(3)] $\cl(S)=S\cup\lp(S)$.
    \end{itemize}
\end{thm}
\begin{pf}
    \begin{itemize}
        \item[(1)] ``$\Longrightarrow$'': $\because x\in\lp(S)$, $\therefore\exists x_n\neq x$, s.t. $x_n\in B\left(x,\frac{1}{n}\right)\cap S$\\
        $\Longrightarrow d(x_n,x)<\frac{1}{n}$, 故 $\exists(x_n)\in S$, s.t. $\forall n$, $x_n\neq x$, $x_n\rightarrow x$.

        ``$\Longleftarrow$'': 设 $(x_n)\rightarrow x$ 且 $x\neq x_n\in S$.\\
        $\forall r>0$, $\exists N$, s.t. 当 $n>N$ 时, $x_n\in B(x,r)$, 即 $\exists x_n\neq x$, s.t. $x_n\in B(x,r)\cap S$, 故 $x\in\lp(S)$.

        综上, 得证.
        \item[(2)] ``$\Longrightarrow$'': 设 $x\in\lp(S)$, 则由 (1) 得, $\exists(x_n)\in S$, s.t. $x_n\neq x\forall n$ 且 $x_n\rightarrow x$.\\
        又 $\because S$ 闭, $\therefore x\in S\Longrightarrow\lp(S)\subseteq S$.

        ``$\Longleftarrow$'': $\forall S$ 中序列 $(x_n)\rightarrow x$, 若 $\exists n$, s.t. $x_n=x$, 则 $x=x_n\in S$,\\
        或 $x_n\neq x\forall n$, 则由 (1) 得 $x\in\lp(S)\subseteq S$, 故 $S$ 闭.

        综上, 得证.
        \item[(3)] 显然 $S\subseteq S\cup\lp(S)\equiv T$.\\
        设 $x\in\lp(T)$, 则由 (1) 得, $\exists$ 序列 $(x_n)\in T$, s.t. $x_n\neq x\forall n$ 且 $x_n\rightarrow x$.\\
        假设 $x\notin S$ 且 $x\notin\lp(S)$, 则 $\because x\notin\lp(S)$, $\therefore\exists r>0$, $B(x,r)\cap S=\emptyset$.\\
        但 $\because(x_n)\rightarrow x$, $\therefore\exists x_n\in B(x,r)\Longrightarrow x_n\notin S$.\\
        又 $\because x_n\in T\equiv S\cup\lp(S)$, $\therefore x_n\in\lp(S)$.\\
        取 $x_n$, s.t. $d(x_n,x)<r$, 则 $B\left(x_n,\frac{r-d(x_n,x)}{2}\right)\subseteq B(x,r)$,\\
        且 $\because x_n\in\lp(S)$, $\therefore\exists y\in S\cap B\left(x_n,\frac{r-d(x_n-x)}{2}\right)\subseteq S\cap B(x,r)$, 与 $B(x,r)\cap S=\emptyset$ 矛盾, 故假设错误, $x\in S$ 或 $x\in\lp(S)$, 即 $x\in T\equiv S\cup \lp(S)\Longrightarrow T$ 闭.\\
        又 $\because S\subseteq T$, $\therefore$ 由闭包定义, $\cl(S)\subseteq T$.

        另一方面, 由闭包定义, $S\in\cl(S)$.\\
        假设 $\lp(S)\nsubseteq\cl(S)$, 即 $\exists x\in\lp(S)$ 且 $x\notin\cl(S)$, 则 $\forall r>0$, $B(x,r)\cap S\neq\emptyset$, 且 $\exists$ 闭集 $S'$, s.t. $x\notin S'$ 即 $x\in(S')^c$.\\
        $\because S'$ 闭 $\Longrightarrow(S')^c$ 开, $\therefore\exists r>0$, s.t. $B(x,r)\subseteq(S')^c$, 与 $B(x,r)\cap S\neq\emptyset$ 矛盾, 故假设错误, $\lp(S)\subseteq\cl(S)\Longrightarrow T\equiv S\cup\lp(S)\subseteq\cl(S)$.

        综上, 得证.
    \end{itemize}
\end{pf}

\section{稠密子集}
\begin{df}[稠密子集]
    $S\subseteq M$, 若 $\cl(S)=M$, 则称 $S$ 为 $M$ 的\textbf{稠密子集}.
\end{df}

若 $S$ 为 $M$ 的稠密子集, 则 $M=\cl(S)=S\cup\lp(S)$, 这意味着 $M$ 中任一点均可由 $S$ 中的某一序列逼近.

\begin{eg}
    $\mathbb{R}=\mathbb{Q}\cup$无理数. $\forall$ 无理数 $r$, $\exists$ 有理数序列 $(r_n)\rightarrow r$, $\therefore\mathbb{Q}$ 为 $R$ 的稠密子集.\\
    同理, 无理数也为 $\mathbb{R}$ 的稠密子集.\\
    实际上, $\mathbb{Q}$ 在 $[0,1]$ 上的测度 $=0$, 即无理数远多于有理数.
\end{eg}

\section{连续}
\begin{df}[连续和不连续]
    度量空间 $(M,d)$ 和 $(M',d')$, 映射 $f:M\rightarrow M'$, $x_0\in M$, 若 $\forall\epsilon>0$, $\exists\delta>0$, s.t. $f(B(x_0,\delta))\subseteq B(f(x_0),\epsilon)$, 即 $d(x,x_0)<\delta\Longrightarrow d'(f(x),f(x_0))<\epsilon$, 则称 $f$ 在 $x_0$ 处\textbf{连续},\\
    若 $\exists\epsilon>0$, $\forall\delta>0$, $f(B(x_0,\delta))\nsubseteq B(f(x_0),\epsilon)$, 则称 $f$ 在 $x_0$ 处\textbf{不连续}.
\end{df}

\begin{thm}[连续的判定 (课本定理 12.4)]
    $f:M\rightarrow M'$ 连续 $\Longleftrightarrow$ 若 $M$ 中序列 $(x_n)$ 收敛于 $x$, 则 $(f(x_n))\rightarrow f(x)$ (即 $f$ 保持收敛性不变).
\end{thm}
\begin{pf}
    ``$\Longrightarrow$'': $\because f$ 连续, $\therefore\forall\epsilon>0$, $\exists\delta>0$, s.t. $f(B(x,\delta))\subseteq B(f(x),\epsilon)$.\\
    $\because(x_n)\rightarrow x$, $\therefore\forall\delta>0$, $\exists N>0$, 当 $n>N$ 时, $d(x_n,x)<\delta$\\
    $\Longrightarrow d(f(x_n),f(x))<\epsilon$, 故 $(f(x_n))\rightarrow f(x)$.

    ``$\Longleftarrow$'': 假设 $f$ 在 $x\in M$ 处不连续, 则 $\exists\epsilon>0$, s.t. $\forall\delta>0$, $f(B(x,\delta))\nsubseteq B(f(x),\epsilon)$, 即 $\exists x'\in B(x,\delta)$, s.t. $f(x')\notin B(f(x),\epsilon)$.\\
    特别地, 取 $\delta=1$, $\exists x_1\in B(x,1)$, s.t. $f(x_1)\notin B(f(x),\epsilon)$,\\
    $\cdots$, 取 $\delta=\frac{1}{n}$, $\exists x_n\in B(x,\frac{1}{n})$, s.t. $f(x_n)\notin B(f(x),\epsilon)$,\\
    $\cdots$, 从而得到序列 $(x_n)\rightarrow x$, 但 $d(f(x_n),f(x))>\epsilon$\\
    $\Longrightarrow f(x)$ 不收敛至 $f(x)$, 与题设矛盾, 故假设错误, $f$ 在 $x_0$ 处连续.

    综上, 得证.
\end{pf}

\begin{thm}[(课本定理 12.5)]
    若 $(x_n)\rightarrow x$, $(y_n)\rightarrow y$, 则 $(d(x_n,y_n))\rightarrow d(x,y)$.
\end{thm}
\begin{pf}
    $\forall\epsilon>0$, $\because(x_n)\rightarrow x$, $\therefore\exists N_1>0$, s.t. 当 $n>N_1$ 时, $d(x_n,x)<\frac{\epsilon}{2}$,\\
    $\because(y_n)\rightarrow y$, $\therefore\exists N_2>0$, s.t. 当 $n>N_2$ 时, $d(y_n,y)<\frac{\epsilon}{2}$\\
    $\Longrightarrow\exists N=\max\{N_1,N_2\}$, s.t. 当 $n>N$ 时, $\abs{d(x_n,y_n)-d(x,y)}=\abs{d(x_n,y_n)-d(x,y_n)+d(x,y_n)-d(x,y)}\leq\abs{d(x_n,y_n)-d(x,y_n)}+\abs{d(x,y_n)-d(x,y)}\leq d(x_n,x)+d(y_n,y)<\epsilon$, 故得证.
\end{pf}

\textbf{推论}: $(d(x_n,y))\rightarrow d(x_0,y)$, 即 $d(,y):M\rightarrow\mathbb{R}$, $x\mapsto d(x,y)$ 为 $M$ 到 $\mathbb{R}$ 的连续映射.\\
同理, $d(,):M\times M\rightarrow\mathbb{R}$, $(x,y)\mapsto d(x,y)$ 亦为连续映射.

\begin{df}[柯西序列]
    $(x_n)$ 为 $M$ 中序列, 若 $\forall\epsilon>0$, $\exists N>0$, 当 $n,m>N$ 时, $d(x_n,x_m)<\epsilon$, 则称 $(x_n)$ 为\textbf{柯西序列}.
\end{df}

\begin{thm}
    收敛 $\Longrightarrow$ 柯西, 反之不真.
\end{thm}
\begin{pf}
    设 $(x_n)\rightarrow x$, 则 $\forall\epsilon>0$, $\exists N_1>0$, s.t. 当 $n>N_1$ 时, $d(x_n,x)<\frac{\epsilon}{2}$,\\
    $\exists N_2>0$, s.t. 当 $m>N_2$ 时, $d(x_m,x)<\frac{\epsilon}{2}$\\
    $\Longrightarrow\exists N=\max\{N_1,N_2\}$, s.t. 当 $n,m>N$ 时, $d(x_n,x_m)\leq d(x_n,x)+d(x_m,x)<\epsilon$, 故 $(x_n)$ 柯西.
\end{pf}

\begin{eg}[不收敛的柯西序列 (课本例 12.12)]
    $C[0,1]\equiv\{[0,1]\text{ 区间上的连续函数}\}$.\\
    $f(x),g(x)\in C[0,1]$, 度量 $d(f(x),g(x))=\int_0^1\abs{f(x)-g(x)}\,\mathrm{d}x$.\\
    令 $f_n(x)=\left\{\begin{array}{ll}
        0,&0\leq x<\frac{1}{2},\\
        n(x-\frac{1}{2}),&\frac{1}{2}\leq x\leq\frac{1}{2}+\frac{1}{n},\\
        1,&\frac{1}{2}+\frac{1}{n}<x\leq 1,
    \end{array}\right.$ 则 $(f_n(x))\rightarrow\left\{\begin{array}{ll}
        0,&0\leq x<\frac{1}{2},\\
        1,&\frac{1}{2}<x<1
    \end{array}\right.$ 不连续, 故 $(f_n(x))$ 柯西但在 $C[0,1]$ 上不收敛 (极限在 $C[0,1]$ 外).
\end{eg}

\section{完备}
\begin{df}[完备]
    称柯西序列均收敛的空间为\textbf{完备}的.
\end{df}

\begin{df}[完备子空间]
    $S$ 为度量空间 $M$ 的子集, 若 $S$ 完备, 则称 $S$ 为 $M$ 的\textbf{完备子空间}.
\end{df}

\begin{thm}[(课本定理 12.6)]\label{thm-12.6}
    对度量空间 $M$,
    \begin{itemize}
        \item[(1)] 任一完备子集闭.
        \item[(2)] 若 $M$ 完备, $S\subseteq M$, 则 $S$ 闭 $\Longleftrightarrow S$ 完备.
    \end{itemize}
\end{thm}
\begin{pf}
    \begin{itemize}
        \item[(1)] 取完备子集 $S\subseteq M$, 取 $S$ 中任意序列 $(x_n)\rightarrow x\in M\Longrightarrow(x_n)$ 柯西.\\
        又 $\because S$ 为完备子集, $\therefore(x_n)$ 收敛, 设 $(x_n)\rightarrow y\in S$.\\
        又 $\because$ 极限唯一, $\therefore x=y\in S\Longrightarrow S$ 闭.
        \item[(2)] ``$\Longleftarrow$'': 已由 (1) 证.

        ``$\Longrightarrow$'': $\forall$ 柯西列 $(x_n)\in S$, $\because S\subseteq M$, $\therefore(x_n)$ 为 $M$ 中的柯西列.\\
        又 $\because M$ 完备, $\therefore(x_n)$ 收敛, 设 $(x_n)\rightarrow x$.\\
        又 $\because S$ 闭, $\therefore x\in S$, 故 $S$ 完备.

        综上, 得证.
    \end{itemize}
\end{pf}

\begin{eg}
    在欧氏度量 $d(u,v)=\sqrt{\sum_i\abs{u_i-v_i}^2}$ 下, $\mathbb{R}$, $\mathbb{C}$, $\mathbb{R}^n$, $\mathbb{C}^n$ 完备.
\end{eg}

\begin{eg}[(课本例 12.11)]
    $C[a,b]$ 上度量 $d(f,g)=\sup_{x\in[a,b]}\{\abs{f(x)-g(x)}\}$, $(C[a,b],d)$ 完备.
    \begin{pf}
        在 $(C[a,b],d)$ 上, $d(f,g)=\sup\{\abs{f(x)-g(x)}\}<\epsilon\Longleftrightarrow\abs{f(x)-g(x)}<\epsilon$, $\forall x\in[a,b]$.\\
        设 $(f_n)$ 柯西, 则 $\forall\epsilon>0$, $\exists N>0$, s.t. 当 $m,n>N$ 时, $d(f_n(x),f_m(x))<\epsilon\Longleftrightarrow\abs{f_n(x)-f_m(x)}<\epsilon$, $\forall x\in[a,b]$,\\
        即给定 $\forall x\in[a,b]$, $(f_n(x))$ 为 $\mathbb{R}$ 或 $\mathbb{C}$ 上的柯西列.\\
        又 $\because\mathbb{R}$ 和 $\mathbb{C}$ 完备, $\therefore(f_n(x))$ 收敛.\\
        设 $f(x)=\lim_{n\rightarrow\infty}(x)$, 则取 $m\rightarrow\infty$ 得当 $n>N$ 时, $\abs{f_n(x)-f(x)}<\epsilon$, $\forall x\in[a,b]$\\
        $\Longrightarrow(f_n)\rightarrow f$, 故 $C[a,b]$ 完备.
    \end{pf}
\end{eg}

\section{等距}
\begin{df}[等距]
    $(M,d)$ 和 $(M',d')$ 为 度量空间, 若映射 $f:M\rightarrow M'$ 满足 $d'(f(x),f(y))=d(x,y)$, 则称 $f$ \textbf{等距}.
\end{df}

\begin{thm}[等距的性质 (课本定理 12.7)]
    $f:(M,d)\rightarrow(M',d')$ 等距, 则
    \begin{itemize}
        \item[(1)] $f$ 单射.
        \item[(2)] $f$ 连续.
        \item[(3)] 若 $f$ 可逆, 则 $f^{-1}$ 等距.
    \end{itemize}
\end{thm}
\begin{pf}
    \begin{itemize}
        \item[(1)] (此处的 $M$ 和 $M'$ 仅为集合, 没有定义额外的运算, 故 $0$ (加法单位元) 不一定存在, 必须从定义证明单射.)\\
        设 $f(x)=f(y)$, 则 $d(f(x),f(y))=0$.\\
        又 $\because f$ 等距, $\therefore d(x,y)=d(f(x),f(y))=0\Longrightarrow x=y$, 故得证.
        \item[(2)] $\forall M$ 的收敛序列 $(x_n)\rightarrow x$, $\forall\epsilon>0$, $\exists N>0$, s.t. 当 $n>N$ 时, $d(x_n,x)<\epsilon$.\\
        $\because f$ 等距, $\therefore d(f(x_n),f(x))=d(x_n,x)<\epsilon\Longrightarrow(f(x_n))\rightarrow f(x)$.\\
        $\because f$ 保持收敛, $\therefore f$ 连续.
        \item[(3)] 若 $f$ 可逆, 则 $f^{-1}(f(x))=x$.\\
        $\because f$ 等距, $\therefore d(f(x),f(y))=d(x,y)=d(f^{-1}(f(x)),f^{-1}(f(y)))\Longrightarrow f^{-1}$ 等距.
    \end{itemize}
\end{pf}

\section{度量空间的完备化}
\begin{thm}[完备化定理 (课本定理 12.8)]
    对度量空间 $(M,d)$, $\exists$ 完备度量空间 $(M',d')$ 及等距 $\tau:M\rightarrow M'$, s.t. $\tau(M)$ 在 $M'$ 中稠密.
\end{thm}

具体如何完备化?\\
取 $V=\{M\text{ 中所有柯西序列}\}$, 定义等价关系 $(x_n)\sim(y_n)\Longleftrightarrow\lim_{n\rightarrow\infty}d(x_n,y_n)=0$, 则等价类 $\frac{V}{\sim}$ 即 $M'$.
\ifx\allfiles\undefined
\end{document}
\fi
